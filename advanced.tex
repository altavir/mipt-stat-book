\chapter{Дополнительные главы}

\small

\section{Принцип максимального правдоподобия и наилучшая оценка среднего}

Пусть при измерениях одно и той же величины два студента
независимым образом получили результаты 
\[
x_{1}\pm\sigma_{1}\qquad\text{и}\qquad x_{2}\pm\sigma_{2}.
\]
Можно ли как-то объединить их ответы и таким образом улучшить оценку
измеряемой величины?

Первое, что может прийти на ум --- найти среднее
арифметическое $\average{x}=\frac{x_{1}+x_{2}}{2}.$ Однако нетрудно
понять, что если, скажем, измерение 2 сильно хуже, чем 1 ($\sigma_{2}\gg\sigma_{1}$),
то разумнее было бы значение $x_{2}$ вообще отбросить и использовать
$x_{1}$ как \textquote{наилучшую} оценку.

Предположим, что измеряемая величина имеет нормальное распределение
с некоторым средним $x_{0}$. Вероятность получить значение $x_{1}$
при первом измерении согласно (\ref{eq:normal}) пропорциональна величине
\[
P_{1}\propto\frac{1}{\sigma_{1}}e^{-\left(x_{1}-x_{0}\right)/2\sigma_{1}^{2}},
\]
вероятность получить значение $x_{2}$ при втором измерении:
\[
P_{2}\propto\frac{1}{\sigma_{2}}e^{-\left(x_{2}-x_{0}\right)^{2}/2\sigma_{2}^{2}}.
\]
Вероятность получить пару значений $\left\{ x_{1},\,x_{2}\right\} $
пропорциональна произведению $P_{1}P_{2}$:
\[
P\left(x_{1},x_{2}\right)=P_{1}P_{2}\propto\frac{1}{\sigma_{1}\sigma_{2}}e^{-\left(x_{1}-x_{0}\right)^{2}/2\sigma_{1}^{2}-\left(x_{2}-x_{0}\right)^{2}/2\sigma_{2}^{2}}.
\]

Рассмотрим выражение, оказавшееся в показателе экспоненты:
\[
\chi^{2}=\left(\frac{x_{1}-x_{0}}{\sigma_{1}}\right)^{2}+\left(\frac{x_{2}-x_{0}}{\sigma_{2}}\right)^{2}.
\]
Назовём \emph{наилучшей} такую оценку параметра
$x_{0}$, при котором полученные в опытах результаты имеют \emph{максимальную
вероятность} ($P\to\mathrm{max}$). Такой подход, называемый
\emph{принципом максимального правдоподобия}, используется
во многих задачах статистики.

Из полученного выше видно, что $P\to\mathrm{max}$, если
величина $\chi^{2}$ (\emph{хи-квадрат}) будет иметь
\emph{минимум}. Дифференцируя по $x_{0}$ и приравнивая
результат к нулю, запишем 
\[
\frac{d\chi^{2}}{dx_{0}}=-2\frac{x_{1}-x_{0}}{\sigma_{1}^{2}}-2\frac{x_{2}-x_{0}}{\sigma_{2}^{2}}=0,
\]
откуда найдём
\begin{equation}
x_{0}=\frac{w_{1}x_{1}+w_{2}x_{2}}{w_{1}+w_{2}},\qquad\text{где }w_{1,2}=\frac{1}{\sigma_{1,2}^{2}}.
\end{equation}

Таким образом, для вычисления \emph{наилучшей} (максимально правдоподобной) оценки
среднего нужно вычислить \emph{взвешенное среднее} с весами, \emph{обратно
пропорциональными квадратам соответствующих погрешностей}.

Результат непосредственно обобщается на произвольное число измерений:
\begin{equation}
x_{\text{наил}}=\frac{\sum\limits _{i}w_{i}x_{i}}{\sum\limits _{i}w_{i}},\qquad w_{i}=\sigma_{i}^{-2}.
\end{equation}

\section{Метод максимального правдоподобия для построения наилучшей
прямой\label{subsec:MMP}}

При описании метода наименьших квадратов мы не обосновали,
почему и в каком смысле именно этот метод является \textquote{наилучшей}
оценкой для коэффициентов линейной регрессии $y=kx+b$. Кроме того,
мы получили формулы только для частного случая, когда погрешности
всех экспериментальных точек одинаковы: $\sigma_{y}=\mathrm{const}$.

Рассмотрим более общий случай. Пусть по-прежнему погрешности
по оси абсцисс малы, $\sigma_{x}\to0$, а погрешности по оси $y$
различны для каждой точки и равны $\sigma_{y_{i}}$. Пусть теория
предсказывает \emph{линейную}\footnote{Отметим, что метод легко обобщается и на нелинейные зависимости общего вида $y=f\left(x;a,b,\ldots\right)$. Хотя формулы
получаются существенно более громоздкими, при вычислении на компьютере
оперировать с ними не сложнее, чем с линейной регрессией. В учебном
практикуме мы рекомендуем всегда делать замену переменных, сводящую
теоретическую зависимость к линейной, поскольку проведение прямой
наиболее наглядно и может быть в грубом приближении проделано просто
\textquote{по линейке}.} 
зависимость $y=kx+b$.

Отклонение точки от теоретической зависимости обозначим как
\[
\Delta y_{i}=y_{i}-\left(kx_{i}+b\right).
\]

Воспользуемся \emph{принципом максимального правдоподобия}
и построим такую прямую, чтобы вероятность обнаружить наблюдаемые
в опыте отклонения $\left\{ \Delta y_{i}\right\} $ от неё была максимальна.

Обозначим вероятность отклонения на величину $\Delta y_{i}$
при известном $\sigma_{y_{i}}$ как $P\!\left(\Delta y_{i};\sigma_{y_{i}}\right)$.
Предположим, что ошибки измерения для всех экспериментальных точек
можно считать \emph{случайными} и \emph{независимыми}.
В таком случае вероятность отклонения для всех $n$ точек равна произведению
вероятностей, так что метод максимального правдоподобия сводится к
поиску максимума выражения
\begin{equation}
\prod\limits _{i=1}^{n}P\!\left(\Delta y_{i};\sigma_{y_{i}}\right)\to\mathrm{max}.\label{eq:MMP_general}
\end{equation}
Максимизация производится по параметрам аппроксимирующей функции (на
нашем случае это $k$ и $b$).

Рассмотрим частный случай, когда погрешности имеют \emph{нормальное}
(гауссово) распределение (\ref{eq:normal}) (напомним, что нормальное
распределение применимо, если отклонения возникают из-за большого
числа независимых факторов, что на практике реализуется довольно часто).
Тогда, поскольку гауссова функция распределения пропорциональна величине
$\propto e^{-\Delta y^{2}/2\sigma^{2}}$, выражение (\ref{eq:MMP_general})
достигает максимума, если минимальна сумма
\begin{equation}
\boxed{\chi^{2}=\sum_{i=1}^{n}\frac{\Delta y_{i}^{2}}{\sigma_{y_{i}}^{2}}\to\mathrm{min}}.\label{eq:chi2}
\end{equation}
Здесь мы ввели стандартное обозначение для такой суммы ---
$\chi^{2}$ (\emph{хи-квадрат)}.

Таким образом, задача построения наилучшей прямой сводится
к минимизации суммы квадратов отклонений, нормированных на соответствующие
дисперсии $\sigma_{y_{i}}^{2}$. Если все погрешности одинаковы, $\sigma_{y_{i}}=\mathrm{const}$,
мы приходим к методу наименьших квадратов.

Получим выражения для наилучших коэффициентов $k$ и $b$.
Заметим, что сумма (\ref{eq:chi2}) является \emph{взвешенной}
суммой квадратов отклонений с весами
\begin{equation}
w_{i}=\frac{1}{\sigma_{y_{i}}^{2}}.
\end{equation}

Можно определить \emph{взвешенное среднее} от
некоторого набора значений $\left\{ x_{i}\right\}$ как
\[
\left\langle x\right\rangle ^{\prime}=\frac{1}{W}\sum_{i}w_{i}x_{i},
\]
где $W=\sum\limits _{i}w_{i}$ --- нормировочная константа.
Далее в этом разделе штрих будем для краткости опускать.

Потребуем, согласно (\ref{eq:chi2}), чтобы была минимальна
сумма
\[
\sum\limits _{i=1}^{n}w_{i}\Delta y_{i}^{2}\to\mathrm{min}.
\]
Повторяя процедуру, использованную при выводе (\ref{eq:MNK}), можно
получить совершенно аналогичные формулы для оптимальных коэффициентов:
\begin{equation}
\boxed{k=\frac{\left\langle xy\right\rangle -\left\langle x\right\rangle \left\langle y\right\rangle }{\left\langle x^{2}\right\rangle -\left\langle x\right\rangle ^{2}},\qquad b=\left\langle y\right\rangle -k\left\langle x\right\rangle },\label{eq:MMP}
\end{equation}
с тем отличием, что под угловыми скобками $\left\langle \ldots\right\rangle $
теперь надо понимать усреднение с весами $w_{i}=1/\sigma_{y_{i}}^{2}$.

Найденные формулы позволяют вычислить коэффициенты линейной
регрессии, \emph{если} известны величины $\sigma_{y_{i}}$.
Значения $\sigma_{y_{i}}$ могут быть получены либо из некоторой теории,
либо измерены непосредственно (многократным повторением измерений
при каждом $x_{i}$), либо оценены из каких-то дополнительных соображений
(например, как инструментальная погрешность).

\paragraph{Погрешности коэффициентов.}

Проведём подробный вывод для погрешностей коэффициентов наилучшей
прямой $\sigma_{k}$ и $\sigma_{b}$. Воспользуемся общей формулой
(\ref{eq:sigma_general}) для погрешности косвенных измерений. Считая,
что величины $x_{i}$ известны точно, запишем для погрешности углового
коэффициента
\[
\sigma_{k}^{2}=\sum\limits _{i}\left(\frac{\partial k}{\partial y_{i}}\right)^{2}\sigma_{y_{i}}^{2}.
\]
Продифференцируем (\ref{eq:MMP}) по $y_{i}$:
\[
\frac{\partial k}{\partial y_{i}}=\frac{1}{D_{xx}}\frac{\partial}{\partial y_{i}}\left(\frac{1}{W}\sum w_{i}x_{i}y_{i}-\left\langle x\right\rangle \frac{1}{W}\sum w_{i}y_{i}\right)=\frac{w_{i}\left(x_{i}-\left\langle x\right\rangle \right)}{WD_{xx}},
\]
где $D_{xx}=\left\langle x^{2}\right\rangle -\left\langle x\right\rangle ^{2}=\left\langle (x-\left\langle x\right\rangle )^{2}\right\rangle $.
Тогда
\[
\sigma_{k}^{2}=\frac{1}{W^{2}D_{xx}^{2}}\sum\limits _{i}w_{i}^{2}\left(x_{i}-\left\langle x\right\rangle \right)^{2}\sigma_{y_{i}}^{2}.
\]
Учитывая, что $w_{i}\sigma_{y_{i}}^{2}=1$, получим
\begin{equation}
\boxed{\sigma_{k}^{2}=\frac{1}{D_{xx}\cdot\sum\limits _{i}\sigma_{y_{i}}^{-2}}}.\label{eq:MMP_sigma_k}
\end{equation}

Аналогично, для погрешности свободного члена имеем
\[
\sigma_{b}^{2}=\sum_{i}\left(\frac{\partial b}{\partial y_{i}}\right)^{2}\sigma_{y_{i}}^{2},
\]
где 
\[
\frac{\partial b}{\partial y_{i}}=\frac{w_{i}}{W}+\frac{\partial k}{\partial y_{i}}\left\langle x\right\rangle =\frac{w_{i}}{W}\left(1-\frac{x_{i}-\left\langle x\right\rangle }{\left\langle x^{2}\right\rangle -\left\langle x\right\rangle ^{2}}\left\langle x\right\rangle \right)=\frac{w_{i}}{W}\frac{\left\langle x^{2}\right\rangle -x_{i}\left\langle x\right\rangle }{D_{xx}}.
\]
Отсюда, пользуясь (\ref{eq:MMP_sigma_k}), приходим к формуле (\ref{eq:MNK_sigma_b}):
\[
\sigma_{b}^{2}=\sigma_{k}^{2}\frac{\left\langle \left(\left\langle x^{2}\right\rangle -x\left\langle x\right\rangle \right)^{2}\right\rangle }{\left\langle x^{2}\right\rangle -\left\langle x\right\rangle ^{2}}=\sigma_{k}^{2}\left\langle x^{2}\right\rangle .
\]

\paragraph{Случай $\sigma_{y}=\mathrm{const}$.}

В частном случае, расмотренном в п. (\ref{subsec:MNK}),
формула (\ref{eq:MMP_sigma_k}) упрощается:
\begin{equation}
\boxed{\sigma_{k}^{2}=\frac{\sigma_{y}^{2}}{nD_{xx}},\qquad\sigma_{b}^{2}=\sigma_{k}^{2}\left\langle x^{2}\right\rangle }.\label{eq:MMP_sigma_k_simple}
\end{equation}
Здесь величина $\sigma_{y}$ может быть оценена непосредственно из
экспериментальных данных:
\begin{equation}
\sigma_{y}\approx\sqrt{\frac{1}{n-2}\sum_{i}\Delta y_{i}^{2}},\label{eq:MMP_sigma_y}
\end{equation}
где $n-2$ --- число \textquote{степеней свободы}
для приращений $\Delta y_{i}$ ($n$ точек за вычетом двух связей
(\ref{eq:MMP})).

Формул (\ref{eq:MMP_sigma_k_simple}) и (\ref{eq:MMP_sigma_y}),
вообще говоря, достаточно для вычисления погрешности величины $k$
по известным экспериментальным точкам. Однако часто их объединяют
в одно упрощённое выражение. Для этого преобразуем (\ref{eq:MMP_sigma_y})
следующим образом: учитывая, что $\left\langle y\right\rangle =k\left\langle x\right\rangle +b$,
запишем 
\[
\Delta y_{i}=y_{i}-kx_{i}-b=\left(y_{i}-\left\langle y\right\rangle \right)-k\left(x_{i}-\left\langle x\right\rangle \right).
\]
Возведём в квадрат, усредним и воспользуемся выражением для $k$ в
форме (\ref{eq:MNK_short}):
\[
\left\langle \Delta y^{2}\right\rangle =D_{yy}+k^{2}D_{xx}-2kD_{xy}=D_{yy}-k^{2}D_{xx}.
\]
Таким образом, 
\[
\sigma_{y}=\sqrt{\frac{n}{n-2}\left(D_{yy}-k^{2}D_{xx}\right)},
\]
и с помощью (\ref{eq:MMP_sigma_k_simple}) получаем формулы (\ref{eq:MNK_sigma}),
(\ref{eq:MNK_sigma_b}):
\[
\boxed{\sigma_{k}=\sqrt{\frac{1}{n-2}\left(\frac{D_{yy}}{D_{xx}}-k^{2}\right)},\qquad\sigma_{b}=\sigma_{k}\sqrt{\left\langle x^{2}\right\rangle }}.
\]

\section{{\small{}Проверка гипотез}}

Предположим, что теория предсказывает некоторую зависимость
\[
y=f\!\left(x;a,b,\ldots\right),
\]
а в эксперименте получен набор значений $\left\{ x_{i},\,y_{i}\right\} $.
Метод максимального правдоподобия позволяет получить параметры $\left\{ a,b,\ldots\right\} $
функции $f$, \textquote{наилучшим} образом приближающие
экспериментальные значения. Причём, как бы плохо экспериментальные
точки не ложились на теоретическую кривую, ответ будет получен в любом
случае. Как проверить, действительно ли измеряемые величины можно
считать связанными зависимостью $y=f\!\left(x\right)$?

Задача может быть решена, если, как обычно, сделать ряд упрощающих
предположений\footnote{Отметим, что в общем случае такая проверка не осуществима:
в частности, если погрешности экспериментальных точек велики или не
известны, то через них с равным \textquote{успехом}
можно провести почти \emph{любую} функцию!}. 
Пусть опять все ошибки измерения \emph{независимы},
распределены \emph{нормально} и нам известны их
среднеквадратичные значения $\sigma_{y_{i}}$ 
(или хотя бы их грубые оценки).

Рассмотрим определённую выше сумму \emph{хи-квадрат} (\ref{eq:chi2}) 
как функцию $n$ переменных $\left\{ y_{i}\right\}$:
\begin{equation}
\chi^{2}\!\left(y_{1},\,y_{2},\,\ldots,\,y_{n}\right)=\sum\limits _{i=1}^{n}\left(\frac{y_{i}-f\!\left(x_{i}\right)}{\sigma_{y_{i}}}\right)^{2}.\label{eq:chi2-1}
\end{equation}
Пусть функция $f\!\left(x\right)$ содержит $p$ \textquote{подгоночных}
параметров (например, $p=2$ для линейной зависимости $f\!\left(x\right)=kx+b$).
Найдём их наилучшие значения по методу максимального правдоподобия
($\chi^{2}\to\mathrm{min}$), и зафиксируем их. После этого $\chi^{2}$
можно рассматривать как функцию 
\[
m=n-p
\]
независимых переменных. Величину $m$ назовём \emph{числом степеней свободы} задачи.

Попробуем сперва качественно ответить на вопрос, какое значение
величины $\chi^{2}$ можно ожидать, если зависимость $y=f\!\left(x\right)$
справедлива? Ясно, что если распределение ошибок нормальное, можно
ожидать отклонений порядка среднеквадратичного: $\Delta y_{i}=y_{i}-f\!\left(x_{i}\right)\sim\sigma_{y_{i}}$.
Поэтому значение суммы (\ref{eq:chi2-1}) должно оказаться порядка
числа входящих в неё независимых слагаемых: $\chi_{m}^{2}\sim m$.
В теории вероятностей доказывается (см., например, \cite{hudson}), что ожидаемое среднее значение (математическое ожидание) $\chi^{2}$ в точности равно числу степеней свободы: $\average{\chi_{m}^{2}}=m$.

Теперь можно сформулировать качественный критерий проверки
гипотезы о наличии некоторой функциональной зависимости (его называют
\emph{критерий хи-квадрат}:
\begin{itemize}
\item  если $\chi^{2}\sim m$, согласие эксперимента с теорией \emph{удовлетворительное}
(гипотеза не опровергнута); 
\item если $\chi^{2}\gg m$ --- \emph{согласия нет}, 
то есть гипотеза о зависимости $y=f\!\left(x\right)$ скорее всего не верна.
\end{itemize}
Заметим, что если вдруг $\chi^{2}\ll m$, то совпадение \emph{слишком}
хорошее, и скорее всего имеет место завышенная оценка для случайных
погрешностей измерения $\sigma_{y_{i}}$.

Для того, чтобы дать строгий \emph{количественный}
критерий, с какой долей вероятности гипотезу $y=f\!\left(x\right)$
можно считать подтверждённой или опровергнутой, нужно детально исследовать
вероятностный закон, которому подчиняется функция $\chi^{2}$. В теории
вероятностей он называется \emph{распределение хи-квадрат}
(с $m$ степенями свободы). В элементарных функциях распределение
хи-квадрат не выражается, но может быть легко найдено численно: функция
встроена во все основные статистические пакеты, либо может быть вычислена
по таблицам. Как правило, определяется вероятность 
$P\left(\chi^{2}>\chi_{0}^{2}\right)$
того, что хи-квадрат имеет значение больше некоторого $\chi_{0}^{2}$,
вычисленного из эксперимента. Если эта вероятность достаточно мала
(например, $P<5\%$; конкретная величина доверительной вероятности
всегда остаётся на усмотрение исследователя), соответствующую гипотезу
следует признать несостоятельной.

{\footnotesize
\textbf{Пример.} Для данных на рис.~\ref{fig:correct}
при $\sigma_{y}=0{,}2$ см величина хи-квадрат равна $\chi^{2}\approx4{,}7$.
За вычетом двух параметров линейной аппроксимации имеем $m=6-2=4$
степеней свободы. По графику TODO определяем, что вероятность того,
что согласие окажется хуже, чем на TODO (т.е. разброс точек относительно
\emph{той же} прямой будет больше),
составляет $P\sim30\%$. Оснований для отказа от <<гипотезы
о линейной зависимости>> нет.

Если бы погрешность каждой точки была равна $\sigma_{y}=0{,}14$
см, то мы получили бы $\chi^{2}=9{,}7$. Это соответствует $P<5\%$,
так что считать зависимость линейной, по-видимому, было бы нельзя
(либо не верна оценка для погрешности $\sigma_{y}$).\par
}%\footnotesize

Напоследок еще раз подчеркнём, что критерий хи-квадрат, во-первых,
статистический и не может дать однозначного ответа --- только
вероятностную оценку, а во-вторых, он работает корректно при условии,
что ошибки разных точек независимы и каждая имеет нормальное распределение.
Эти предположения, вообще говоря, выполняются далеко не всегда и,
по-хорошему, требуют отдельной проверки.