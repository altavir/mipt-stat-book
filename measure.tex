\chapter{Измерения и погрешности}

% \disclaimer{
%     Этот раздел предназначен для поверхностного ознакомления с измерением
% физических величин. Для более глубокого понимания рекомендуется сначала
% ознакомиться с основами теории вероятности, приведенными в главе \ref{ch:prob}.
% }

Свойства физического объекта (явления, процесса) определяются набором
количественных характеристик --- \emph{физических величин}.
Как правило, результат измерения представляет
собой \emph{число}, задающее отношение измеряемой величины к некоторому \emph{эталону}.
Сравнение с эталоном может быть как
\emph{прямым} (проводится непосредственно
экспериментатором), так и \emph{косвенным} (проводится с помощью некоторого
прибора, которому экспериментатор доверяет).
Полученные таким образом величины имеют \emph{размерность}, определяемую выбором эталона.

\note{Результатом измерения может также служить количество отсчётов некоторого
события, логическое утверждение (да/нет) или даже качественная оценка
(сильно/слабо/умеренно). Мы ограничимся наиболее типичным для физики случаем,
когда результат измерения может быть представлен в виде \emph{числа} или \emph{набора чисел}.}

Взаимосвязь между различными физическими величинами может быть описана
\emph{физическими законами}, представляющими собой идеализированную
\emph{модель} действительности. Конечной целью любого физического
эксперимента (в том числе и учебного) является проверка адекватности или
уточнение параметров таких моделей.

\section{Результат измерения}

Рассмотрим простейший пример: измерение длины стержня
с помощью линейки. Линейка проградуирована производителем с помощью
некоторого эталона длины --- таким образом, сравнивая длину
стержня с ценой деления линейки, мы выполняем косвенное сравнение с
общепринятым стандартным эталоном.

% Важно понимать, что результат измерения не отражает точную (абсолютную)
% характеристику объекта, но в зависимости от самого измерение является числом
% \emph{приблизительно} равным этому абсолютному значению.

Допустим, мы приложили линейку к стержню и увидели на шкале некоторый результат
$x=x_{\text{изм}}$. Можно ли утверждать, что $x_{\text{изм}}$ --- это длина
стержня?

Во-первых, значение $x$ \emph{не может быть задано точно}, хотя бы
потому, что оно обязательно \emph{округлено} до некоторой значащей
цифры: если линейка \textquote{обычная}, то у неё
есть \emph{цена деления}; а если линейка, к примеру, \textquote{лазерная}
--- у неё высвечивается \emph{конечное число значащих цифр}
на дисплее.

Во-вторых, мы никак не можем быть уверенны, что длина стержня \emph{на
самом деле} такова хотя бы с точностью до ошибки округления. Действительно,
мы могли приложить линейку не вполне ровно; сама линейка могла быть
изготовлена не вполне точно; стержень может быть не идеально цилиндрическим
и т.п.

И, наконец, если пытаться хотя бы гипотетически переходить к бесконечной
точности измерения, \emph{теряет смысл само понятие} \textquote{длины стержня}. Ведь
на масштабах атомов у стержня нет чётких границ, а значит говорить о его
геометрических размерах в таком случае крайне затруднительно!

Итак, из нашего примера видно, что никакое физическое измерение не может быть
произведено абсолютно точно, то есть
\textbf{у любого измерения есть \emph{погрешность}}.%
\note{Также используют эквивалентный термин \emph{ошибка измерения}
(от \emph{англ.} error). Подчеркнём, что смысл этого термина отличается от
общеупотребительного бытового: если физик говорит \textquote{в измерении есть ошибка},
--- это не означает, что оно неправильно и его надо переделать.
Имеется ввиду лишь, что это измерение \emph{неточно}, то есть имеет
\emph{погрешность}.}

Количественно погрешность можно было бы определить как разность между
измеренным и \textquote{истинным} значением длины стержня:
$\delta x=x_{\text{изм}}-x_{\text{ист}}$. Однако на практике такое определение
использовать нельзя: во-первых, из-за неизбежного наличия
погрешностей \textquote{истинное} значение измерить невозможно, и во-вторых, само
\textquote{истинное} значение может отличаться в разных измерениях (например, стержень
неровный или изогнутый, его торцы дрожат из-за тепловых флуктуаций и т.д.).
Поэтому говорят обычно об \emph{оценке} погрешности.

Об измеренной величине также часто говорят как об \emph{оценке}, подчеркивая,
что эта величина не точна и зависит не только от физических свойства
исследуемого объекта, но и от процедуры измерения.

\note{
    Термин \emph{оценка} имеет и более формальное значение. Оценкой называют результат процедуры получения значения параметра или параметров физической модели, а также иногда саму процедуру. Теория оценок является подразделом математической статистики. Некоторые ее положения изложены в главе \ref{ch:estimate}, но для более серьезного понимания следует обратиться к \cite{idie}.
}

Для оценки значения физической величины корректно использовать
не просто некоторое фиксированное число $x_{\text{изм}}$, а \emph{интервал} (или
\emph{диапазон}) значений, в пределах которого может лежать её
\textquote{истинное} значение. В простейшем случае этот интервал
может быть записан как
\[
x=x_{\text{изм}}\pm\delta x,
\]
где $\delta x$ --- \emph{абсолютная} величина погрешности.
Эта запись означает, что исследуемая величина лежит в интервале
$x\in(x_{\text{изм}}-\delta x;\,x_{\text{изм}}+\delta x)$
с некоторой достаточно большой долей вероятности (более подробно о
вероятностном содержании интервалов см. п.~\ref{ch:estimate}).
Для наглядной оценки точности измерения удобно также использовать
\emph{относительную} величину погрешности:
\[
\varepsilon_{x}=\frac{\delta x}{x_{\text{изм}}}.
\]
Она показывает, насколько погрешность мала по сравнению с
самой измеряемой величиной (её также можно выразить в процентах:
$\varepsilon=\frac{\delta x}{x}\cdot100\%$).

\example{Штангенциркуль ---
прибор для измерения длин с ценой деления $0{,}1\;\text{мм}$. Пусть
диаметр некоторой проволоки равен $0{,}37$~мм. Считая, что абсолютная
ошибка составляет половину цены деления прибора, результат измерения
можно будет записать как $d=0{,}40\pm0{,}05\;\text{мм}$ (или
$d=(40\pm5)\cdot10^{-5}\;\text{м}$).
Относительная погрешность составляет $\varepsilon\approx13\%$, то
есть точность измерения весьма посредственная --- поскольку
размер объекта близок к пределу точности прибора.}

\paragraph{О необходимости оценки погрешностей.}

Измерим длины двух стержней $x_{1}$ и $x_{2}$ и сравним результаты.
Можно ли сказать, что стержни одинаковы или различны?

Казалось бы,
достаточно проверить, справедливо ли $x_{1}=x_{2}$. Но \emph{никакие}
два результата измерения не равны друг другу с абсолютной точностью! Таким
образом, без указания погрешности измерения ответ на этот вопрос дать
\emph{невозможно}.

С другой стороны, если погрешность $\delta x$ известна, то можно
утверждать, что если измеренные длины одинаковы
\emph{в пределах погрешности опыта}, если $|x_{2}-x_{1}|<\delta x$
(и различны в противоположном случае).

Итак, без знания погрешностей невозможно сравнить между собой никакие
два измерения, и, следовательно, невозможно сделать \emph{никаких}
значимых выводов по результатам эксперимента: ни о наличии зависимостей
между величинами, ни о практической применимости какой-либо теории,
и т.\,п. В связи с этим задача правильной оценки погрешностей является крайне
важной, поскольку существенное занижение или завышение значения погрешности
(по сравнению с реальной точностью измерений) ведёт к \emph{неправильным выводам}.

В физическом эксперименте (в том числе лабораторном практикуме) оценка
погрешностей должна проводиться \emph{всегда}
(даже когда составители задания забыли упомянуть об этом).

\section{Многократные измерения}

Проведём серию из $n$ \emph{одинаковых} (\emph{однотипных}) измерений одной
и той же физической величины (например, многократно приложим линейку к стержню) и получим
ряд значений
\[
\left\{ x_{1},\,x_{2},\,\ldots\,,x_{n}\right\} .
\]
Что можно сказать о данном наборе чисел и о длине стержня?
И можно ли увеличивая число измерений улучшить конечный результат?

Если цена деления самой линейки достаточно мала, то как нетрудно убедиться
на практике, величины $\left\{ x_{i}\right\}$ почти наверняка окажутся
\emph{различными}. Причиной тому могут быть
самые разные обстоятельства, например: у нас недостаточно остроты
зрения и точности рук, чтобы каждый раз прикладывать линейку одинаково;
стенки стержня могут быть слегка неровными; у стержня может и не быть
определённой длины, например, если в нём возбуждены звуковые волны,
из-за чего его торцы колеблются, и т.\,д.

В такой ситуации результат измерения интерпретируется как
\emph{случайная величина}, описываемая некоторым \emph{вероятностным} законом
(\emph{распределением}).
Подробнее о случайных величинах и методах работы с ними см. гл.~\ref{ch:prob}.

% Как правило предполагают, что распределение, характеризующее изучаемую величину
% при заданном способе измерения (на примере с линейкой легко убедиться, что
% результат зависит не только от изучаемого объекта но и от методики его
% исследования), устроено таким образом, что его среднее значение (математическое
% ожидание) соответствует истинной величине.
% \todo[author=ppv,inline]{А почему так предполагают? И что такое "истинная
% величина"? Надо переформулировать.}
% В этом случае разумно в качестве оценки значения по серии измерений использовать
% среднее арифметическое:

По набору результатов $\left\{x_i\right\}$ можно вычислить их среднее арифметическое:
\begin{equation}
 \average{x}=\frac{x_{1}+x_{2}+\ldots+x_{n}}{n}\equiv\frac{1}{n}\sum\limits
_{i=1}^{n}x_{i}\label{eq:average}.
\end{equation}
Это значение, вычисленное по результатам конечного числа $n$ измерений,
принято называть \emph{выборочным} средним. Здесь и далее для обозначения
выборочных средних будем использовать угловые скобки.

% Для количественной оценки разброса
% используют понятие \emph{дисперсии}. Понятие дисперсии и его формальное
% определение описаны в разделе \ref{ch:prob}. В этой главе ограничимся тем, что
% введем понятие среднеквадратичного отклонения (среднеквадратичное отклонение по
% определению равно корню из дисперсии), и его определение по серии измерений.

Кроме среднего представляет интерес и то, насколько сильно варьируются
результаты от опыта к опыту. Определим отклонение каждого измерения от среднего как
\[
\Delta x_{i}=x_{i}-\average{x},\qquad i=1\ldots n.
\]
Разброс данных относительно среднего принято характеризовать
\emph{среднеквадратичным отклонением}:
\begin{equation}
s_{x}=\sqrt{\frac{\Delta x_{1}^{2}+\Delta x_{2}^{2}+\ldots+\Delta
x_{n}^{2}}{n}}=\sqrt{\frac{1}{n}\sum\limits _{i=1}^{n}\Delta
x_{i}^{2}}\label{eq:sigma}
\end{equation}
или кратко
\begin{equation}
\boxed{s_{x}^{2}=\average{\left(x-\average{x}\right)^{2}}}.\label{eq:sigma_s}
\end{equation}
Значение среднего квадрата отклонения $s_{x}^{2}$ также называют
выборочной \emph{дисперсией}.

% \note{Разброс значений можно характеризовать по-другому, например: средним
%     или максимальным модулем отклонения. Из дальнейшего рассмотрения будет видно,
%     почему использование среднеквадратичного значения математически наиболее удобно.
%
%     Среднее арифметическое является далеко не единственным способом получения оценки
% по выборке (набору данных), в некоторых случаях к примеру используют наиболее
% вероятное значение, но такие случаи лежат за пределами данного пособия.
% }

Будем увеличивать число измерений $n$ ($n\to \infty$). Если объект измерения и методика
достаточно стабильны, то отклонения от среднего $\Delta x_i$ будут, во-первых,
относительно малы, а во-вторых, положительные и отрицательные отклонения будут
встречаться примерно одинаково часто. Тогда при вычислении \eqref{eq:average}
почти все отклонения $\Delta x_i$ скомпенсируются и можно ожидать,
что выборочное среднее при $n\gg 1$ будет стремиться к некоторому пределу:
\[
\lim_{n\to\infty}\frac{1}{n}\sum_{i=1}^{n}x_{i}=x_{0}.
\]
Тогда предельное значение $x_{0}$ можно отождествить с \textquote{истинным} средним
для исследуемой величины.

Предельную величину среднеквадратичного отклонения при $n\to\infty$
обозначим как
\[
\lim\limits _{n\to\infty}s_{x} = \sigma_{x}.
\]

\note{В общем случае указанные пределы могут и не существовать. Например, если измеряемый параметр
меняется во времени, либо в результате самого измерения, либо испытывает слишком большие
случайные скачки и т.\,п. Такие ситуации требуют особого рассмотрения и мы на них не
останавливаемся.}

Итак, можно по крайней мере надеяться на то, что результаты небольшого числа
измерений имеет не слишком большой разброс, так что величина $\average{x}$
может быть использована как приближенное значение (\emph{оценка}) истинного значения
$\average{x}\approx \limaverage{x}$,
а увеличение числа измерений позволит уточнить результат.
Многие случайные величины подчиняются так называемому \emph{нормальному закону}
распределения (подробнее см. раздел~\ref{ch:prob}). Для таких величин
могут быть строго доказаны следующие свойства:
\begin{itemize}
    \item при многократном повторении эксперимента б\'{о}льшая часть измерений
    ($\sim$68\%) попадает в интервал $\limaverage{x}-\sigma_{x}<x<x+\sigma_{x}$
    (см. п.~\ref{sec:normal}).
    \item выборочное среднее значение $\average{x}$ оказывается с большей
    вероятностью ближе к истинному значению $\limaverage{x}$, чем каждое из измерений
    $\left\{x_i\right\}$ в отдельности. При этом ошибка вычисления среднего
    убывает пропорционально корню из числа опытов $\sqrt{n}$
    (см. п.~\ref{sec:average}).
\end{itemize}

% В теории вероятности показывается (см. раздел \ref{ch:prob}), что среднее
% арифметическое значение величин с одинаковым распределением имеет значительно
% меньший разброс, чем единичное измерение.
%
% В математической статистике показывается, что если некоторая величина,
% распределенная по нормальному закону (см. \ref{sec:normal}) имеет среднее
% значение $\average{x}$ и стандартное отклонение $\sigma_x$, то при многократном
% повторении эксперимента приблизительно в 68\% случаев результат $x$ будет
% попадать в интервал $\average{x}-\sigma_{x}<x<\average{x}+\sigma_{x}.$

\exercise{Показать, что
\begin{equation}
% s_{x}^{2}=\average{\left(x-\average{x}\right)^{2}}=\average{x^{2}-2x\average{x}
% +\average{x}^{2}}=\average{x^{2}}-\average{x}^{2}.
s_{x}^{2}=\average{x^{2}}-\average{x}^{2}.
\label{eq:sigma_x2}
\end{equation}
то есть дисперсия равна разности среднего значения квадрата
$\average{x^{2}}=\frac{1}{n}\sum\limits _{i=1}^{n}x_{i}^{2}$
и квадрата среднего $\average{x}^{2}=\left(\frac{1}{n}\sum\limits
_{i=1}^{n}x_{i}\right)^{2}$.}

\section{Классификация погрешностей}

Чтобы лучше разобраться в том, нужно ли многократно повторять измерения,
и в каком случае это позволит улучшить результаты опыта,
проанализируем источники и виды погрешностей.

В первую очередь, многократные измерения позволяют проверить
\emph{воспроизводимость} результатов: повторные измерения в \emph{одинаковых}
условиях, должны давать близкие результаты. В противном случае
исследование будет существенно затруднено, если вообще возможно.
Таким образом, многократные измерения \emph{необходимы} для того,
чтобы убедиться как в надёжности методики, так и в существовании измеряемой
величины как таковой.
% \note{Невоспроизводимые явления исследовать, очевидно, невозможно. Например,
% мы почти ничего не знаем об устройстве \textquote{шаровой молнии},
% поскольку ещё никому не удалось создать стабильные условия для её
% получения, хотя свидетельства существования этого редкого атмосферного
% явления довольно многочисленны и надёжны.}

При любых измерениях возможны грубые ошибки --- \emph{промахи}
(\emph{англ.} miss). Это \textquote{ошибки} в стандартном
понимании этого слова --- возникающие по вине экспериментатора
или в силу других непредвиденных обстоятельств (например, из-за сбоя
аппаратуры). Промахов, конечно, нужно избегать, а результаты таких
измерений должны быть по возможности исключены из рассмотрения.

\note{Порой и \emph{весь эксперимент} может оказаться одним большим
\textquote{промахом}. Из недавних примеров можно вспомнить сверхсветовые нейтрино,
якобы обнаруженные в 2011~г. на установке OPERA --- там причиной \textquote{открытия}
оказался плохой контакт кабеля. Самый радикальный способ проверки достоверности
результатов --- многократная независимая проверка на других
приборах, другими методами, и желательно, другими экспериментаторами.}

Как понять, является ли \textquote{аномальный} результат промахом? Вопрос этот весьма
непрост. В литературе существуют статистические
критерии отбора промахов, которыми мы, однако, настоятельно \emph{не рекомендуем}
пользоваться (по крайней мере, без серьезного понимания последствий
такого отбора). Отбрасывание аномальных данных может, во-первых, привести
к тенденциозному искажению результата исследований, а во-вторых, так
можно упустить открытие неизвестного эффекта. Поэтому при научных
исследованиях необходимо максимально тщательно проанализировать причину
каждого промаха, в частности, многократно повторив эксперимент. Лишь
только если факт и причина промаха установлены вполне достоверно,
соответствующий результат можно отбросить.

\note{Часто причины аномальных отклонений невозможно установить на этапе
обработки данных, поскольку часть информации о проведении измерений к этому моменту
утеряна. Единственным способ борьбы с этим --- это максимально подробное описание всего
процесса измерений в \emph{лабораторном журнале}. Подробнее об этом
см. п.~\ref{sec:journal}.}

При многократном повторении измерении одной и той же физической величины
погрешности могут иметь \emph{систематический} либо \emph{случайный}
характер. Назовём погрешность \emph{систематической}, если она повторяется
от опыта к опыту, сохраняя свой знак и величину, либо \emph{закономерно}
меняется в процессе измерений. \emph{Случайные} (или \emph{статистические})
погрешности меняются хаотично при повторении измерений как по величине,
так и по знаку, и в изменениях не прослеживается какой-либо закономерности.

Кроме того, удобно разделять погрешности по их происхождению. Можно
выделить
\begin{itemize}
    \item \emph{инструментальные} (или \emph{приборные}) \emph{погрешности},
связанные с несовершенством конструкции (неточности, допущенные при
изготовлении или вследствие старения), ошибками калибровки или ненормативными
условиями эксплуатации измерительных приборов;
    \item \emph{методические} \emph{погрешности}, связанные с несовершенством
теоретической модели явления (использование приближенных формул и
моделей явления) или с несовершенством методики измерения (например,
влиянием взаимодействия прибора и объекта измерения на результат измерения);
    \item \emph{естественные} \emph{погрешности}, связанные со случайным
характером
измеряемой физической величины --- они являются не столько
\textquote{ошибками} измерения, сколько характеризуют
природу изучаемого объекта или явления.
\end{itemize}

\note{Разделение погрешностей на систематические и случайные
не является однозначным и зависит от постановки опыта. Например, производя
измерения не одним, а несколькими однотипными приборами, мы переводим
систематическую приборную ошибку, связанную с неточностью шкалы и
калибровки, в случайную. Разделение по происхождению также условно,
поскольку любой прибор подвержен воздействию \textquote{естественных}
случайных и систематических ошибок (шумы и наводки, тряска, атмосферные
условия и т.\,п.), а в основе работы прибора всегда лежит некоторое
физическое явление, описываемое не вполне совершенной теорией.}

\subsection{Случайные погрешности}

Случайный характер присущ большому количеству различных физических
явлений, и в той или иной степени проявляется в работе всех без исключения
приборов. Случайные погрешности обнаруживаются просто при многократном
повторении опыта --- в виде хаотичных изменений (\emph{флуктуаций})
значений $\left\{ x_{i}\right\} $.

Если случайные отклонения от среднего в большую или меньшую стороны
примерно равновероятны, можно рассчитывать, что при вычислении среднего
арифметического (\ref{eq:average}) эти отклонения скомпенсируются,
и погрешность результирующего значения $\average{x}$ будем меньше,
чем погрешность отдельного измерения.

Случайные погрешности бывают связаны, например,
\begin{enumerate}

    \item \emph{с особенностями используемых приборов}: техническими
недостатками
(люфт в механических приспособлениях, сухое трение в креплении стрелки
прибора), с естественными (тепловой и дробовой шумы в электрических
цепях, тепловые флуктуации и колебания измерительных устройств из-за
хаотического движения молекул, космическое излучение) или техногенными
факторами (тряска, электромагнитные помехи и наводки);

    \item \emph{с особенностями и несовершенством методики измерения} (ошибка
при отсчёте по шкале, ошибка времени реакции при измерениях с секундомером);

    \item \emph{с несовершенством объекта измерений} (неровная поверхность,
неоднородность состава);

    \item \emph{cо случайным характером исследуемого явления} (радиоактивный
распад, броуновское движение).

\end{enumerate}

Остановимся несколько подробнее на двух последних случаях. Они отличаются
тем, что случайный разброс данных в них порождён непосредственно объектом
измерения. Если при этом приборные погрешности малы, то \textquote{ошибка}
эксперимента возникает лишь в тот момент, когда мы \emph{по своей
воле} совершаем замену ряда измеренных значений на некоторое среднее
$\left\{ x_{i}\right\} \to\average{x}$. Разброс данных при этом
характеризует не точность измерения, а сам исследуемый объект или
явление. Однако с \emph{математической} точки зрения приборные и
\textquote{естественные}
погрешности \emph{неразличимы} --- глядя на одни только
экспериментальные данные невозможно выяснить, что именно явилось причиной
их флуктуаций: сам объект исследования или иные, внешние причины.
Таким образом, для исследования естественных случайных процессов необходимо
сперва отдельно исследовать и оценить случайные инструментальные погрешности
и убедиться, что они достаточно малы.

% \note{Зависимость случайной погрешности среднего от погрешности отдельного
% измерения как правило имеет вид $\sigma_N = \sigma_1 / \sqrt{N}$ и выводится в
% разделе \ref{sec:average}.
% }

\subsection{Систематические погрешности}

Систематические погрешности, в отличие от случайных, невозможно обнаружить,
исключить или уменьшить просто многократным повторением измерений.
Они могут быть обусловлены, во-первых, неправильной работой приборов
(\emph{инструментальная погрешность}), например, cдвигом нуля отсчёта
по шкале, деформацией шкалы, неправильной калибровкой, искажениями
из-за не нормативных условий эксплуатации, искажениями из-за износа
или деформации деталей прибора, изменением параметров прибора во времени
из-за нагрева и т.п. Во-вторых, их причиной может быть ошибка в интерпретации
результатов (\emph{методическая погрешность}), например, из-за использования
слишком идеализированной физической модели явления, которая не учитывает
некоторые значимые факторы (так, при взвешивании тел малой плотности
в атмосфере необходимо учитывать силу Архимеда; при измерениях в электрических
цепях может быть необходим учет неидеальности амперметров и вольтметров
и т.\,д.).

Систематические погрешности условно можно разделить на следующие категории.

\begin{enumerate}
    \item Известные погрешности, которые могут быть достаточно точно вычислены
или измерены. При необходимости они могут быть учтены непосредственно:
внесением поправок в расчётные формулы или в результаты измерений.
Если они малы, их можно отбросить, чтобы упростить вычисления.

    \item Погрешности известной природы, конкретная величина которых неизвестна,
но максимальное значение вносимой ошибки может быть определено теоретически
или экспериментально. Такие погрешности неизбежно присутствуют в любом
опыте, и задача экспериментатора --- свести их к минимуму,
совершенствуя методики измерения и выбирая более совершенные приборы.

    Чтобы оценить величину систематических погрешностей опыта, необходимо
учесть паспортную точность приборов (производитель, как правило, гарантирует,
что погрешность прибора не превосходит некоторой величины), проанализировать
особенности методики измерения, и по возможности, провести контрольные
опыты.

    \item Погрешности известной природы, оценка величины которых по каким-либо
причинам затруднена (например, сопротивление контактов при подключении
электронных приборов). Такие погрешности должны быть обязательно исключены
посредством модификации методики измерения или замены приборов.

    \item Наконец, нельзя забывать о возможности существования ошибок, о
которых мы не подозреваем, но которые могут существенно искажать результаты
измерений. Такие погрешности самые опасные, а исключить их можно только
многократной \emph{независимой} проверкой измерений, разными методами
и в разных условиях.
\end{enumerate}

В учебном практикуме учёт систематических погрешностей ограничивается,
как правило, паспортными погрешностями приборов и теоретическими поправками
к упрощенной модели исследуемого явления.
