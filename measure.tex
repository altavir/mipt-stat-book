\chapter{Измерения и погрешности}

\disclaimer{
    Этот раздел предназначен для поверхностного ознакомления с измерением физических величин. Для более глубокого понимания рекомендуется сначала ознакомиться с основами теории вероятности, приведенными в главе \ref{ch:prob}.
}

Конечной целью любого физического эксперимента (в том числе и учебного) является проверка или уточнение параметров некоторой математической (или наивной) модели, описывающей действительность. Для этого как правило проводят серию измерений физических величин, при этом результат каждого измерения является числом или набором чисел. Эти числа потом используются для проверки адекватности модели или уточнения ее параметров.

Надо отметить, что результаты измерений не всегда представлены именно числом, это может быть логическое утверждение (да / нет) и даже словом или набором символов. В физике как правило работают с числами, поэтому в этом пособии мы ограничимся исключительно численными величинами.

\section{Результат измерения}

Для получения численной оценки результата измерения для какого-либо физического объекта, явления или процесса, какой-то из его физических параметров сравнивают с некоторым emph{эталоном}. Это сравнение может быть как прямым (проводиться непосредственно экспериментатором), так и косвенным (проводиться при помощи некоторого прибора, которому экспериментатор доверяет).

Для наглядности рассмотрим простейший пример измерения длины стержня
с помощью линейки. Линейка проградуирована производителем с помощью
некоторого эталона длины --- таким образом, сравнивая длину
стержня с ценой деления линейки, мы косвенно сравниваем его с принятым
стандартным эталоном. 

Важно понимать, что результат измерения не отражает точную (абсолютную) характеристику объекта, но в зависимости от самого измерение является числом \emph{приблизительно} равным этому абсолютному значению. 

Допустим, мы приложили линейку к стержню и увидели некоторый результат
$x=x_{\text{изм}}$. Что можно сказать о длине стержня?

Во-первых, значение $x$ \emph{не может быть задано точно}, хотя бы
потому, что оно обязательно \emph{округлено} до некоторой значащей
цифры: если линейка <<обычная>>, то у неё
есть \emph{цена деления}; а если линейка, к примеру, <<лазерная>>
--- у неё высвечивается \emph{конечное число значащих цифр}
на дисплее.

Во-вторых, мы никак не можем быть уверенны, что длина стержня \emph{на
самом деле} такова хотя бы с точностью до ошибки округления. Действительно,
мы могли приложить линейку не вполне ровно; сама линейка могла быть
изготовлена не вполне точно; стержень может быть не идеально цилиндрическим
и т.п.

Итак, из нашего примера видно, что никакое физическое измерение не может быть
произведено абсолютно точно, то есть у любого измерения есть \emph{погрешность}.
\footnote{Также используют эквивалентный термин \emph{ошибка измерения} (от\emph{
англ.} error). Подчеркнём, что смысл этого термина отличается от общеупотребительного
бытового: если физик говорит <<в измерении есть ошибка>>,
--- это не означает, что оно неправильно и его надо переделать.\emph{
}Имеется ввиду лишь, что это измерение \emph{неточно}, то есть имеет
\emph{погрешность}. } 

Количественно погрешность определяется как разность между измеренным и <<истинным>> значением длины стержня: $\delta x=x_{0}-x_{\text{ист}}$. На практике такое определение нельзя использовать, поскольку во-первых, раз у любого измерения есть погрешность, значит <<истинное>> значение измерить невозможно, во-вторых, само <<истинное>> значение может быть не определено вовсе (например, стержень неровный или изогнутый, его торцы дрожат из-за тепловых флуктуаций и т.д.). Поэтому говорят обычно об \emph{оценке} погрешности.

Об измеренной величине также часто говорят как об \emph{оценке}, подчеркивая, что эта величина зависит не только от физических свойства исследуемого объекта, но и от процедуры измерения. Подробно свойства оценок рассмотрены в разделе \ref{sec:estimates}.

В силу неизбежности наличия погрешности, часто более корректным является использование для оценки величины не фиксированного значения, а интервала (или диапазона) значений, в пределах которого может лежать измеряемая величина. В простом случае этот интервал является симметричным относительно оценки величины и может быть записан как
\[
x=x_{\text{изм}}\pm\delta x,
\]
где $\delta x$ --- \emph{абсолютная величина} погрешности.
Эта запись означает, что исследуемая величина лежит в интервале $x\in(x_{\text{изм}}-\delta x;\,x_{\text{изм}}+\delta x)$
с некоторой достаточно большой долей вероятности (более подробно о вероятностном содержании интервалов в разделе \ref{sec:estimates}). Во многих случаях удобно использовать не абсолютную а относительную погрешность:
\[
\varepsilon_{x}=\frac{\delta x}{x_{\text{изм}}},
\]
то есть то, насколько погрешность мала по сравнению с самой измеряемой
величиной (ее можно выразить в процентах: $\varepsilon=\frac{\delta x}{x}\cdot100\%$).

\example{ Штангенциркуль ---
прибор для измерения длин с ценой деления $0{,}1\;\text{мм}$. Пусть
диаметр некоторой проволоки равен $0{,}37$~мм. Считая, что абсолютная
ошибка составляет половину цены деления прибора, результат измерения
можно будет записать как $d=0{,}40\pm0{,}05\;\text{мм}$ (или $d=(40\pm5)\cdot10^{-5}\;\text{м}$).
Относительная погрешность составляет $\varepsilon\approx13\%$, то
есть точность измерения весьма посредственная --- поскольку
размер объекта близок к пределу точности прибора.
}

\paragraph{О необходимости оценки погрешностей.}

Измерим длины двух стержней $x_{1}$ и $x_{2}$ и сравним результаты.
Можно ли сказать, что стержни одинаковы или различны? Казалось бы,
достаточно проверить, справедливо ли $x_{1}=x_{2}$. Но \emph{никакие}
два результата не равны друг другу с абсолютной точностью! Таким образом,
без указания погрешности измерения ответ на этот вопрос дать \emph{невозможно}. 

С другой стороны, если погрешность $\delta x$ известна, то можно
утверждать, что если измеренные длины одинаковы \emph{в пределах погрешности
опыта}, если $|x_{2}-x_{1}|<\delta x$ (и различны в противоположном
случае).

Итак, без знания погрешностей невозможно сравнить между собой никакие
два измерения, и, следовательно, невозможно сделать \emph{никаких}
значимых выводов по результатам эксперимента: ни о наличии зависимостей
между величинами, ни о практической применимости какой-либо теории,
и т.\,п. В связи с этим крайне важной является задача правильной
оценки погрешностей, поскольку её существенное занижение или завышение
(по сравнению с реальной точностью измерений) ведёт к \emph{неправильным
выводам}.

В физическом эксперименте (в том числе лабораторном практикуме) оценка погрешностей должна проводиться \emph{всегда}
(даже когда составители задания забыли упомянуть об этом).

\section{Многократные измерения}

Проведём серию из $n$ \emph{одинаковых (однотипных}) измерений одного
и того же объекта (многократно приложили линейку к стержню) и получим
ряд значений
\[
\left\{ x_{1},\,x_{2},\,\ldots\,,x_{n}\right\} .
\]
Что можно сказать о данном наборе чисел и о длине стержня?

Если цена деления самой линейки достаточно мала, то нетрудно убедиться, что величины $\left\{ x_{i}\right\} $
почти наверняка окажутся \emph{различными}. Причиной тому могут быть
самые разные обстоятельства, например: у нас недостаточно остроты
зрения и точности рук, чтобы каждый раз прикладывать линейку одинаково;
стенки стержня могут быть слегка неровными; у стержня может и не быть
определённой длины, например, если в нём возбуждены звуковые волны,
из-за чего его торцы колеблются, и т.\,д.

В такой ситуации результат измерения интерпретируется как \emph{случайная величина}, описываемая некоторым \emph{распределением}. Подробно определение случайных величин и методы работы с ним описаны в разделе \ref{seq:prob}.

Как правило предполагают, что распределение, характеризующее изучаемую величину при заданном способе измерения (на примере с линейкой легко убедиться, что результат зависит не только от изучаемого объекта но и от методики его исследования), устроено таким образом, что его среднее значение (математическое ожидание) соответствует истинной величине. 
\todo[author=ppv,inline]{А почему так предполагают? И что такое "истинная величина"? Надо переформулировать.}
В этом случае разумно в качестве оценки значения по серии измерений использовать среднее арифметическое:
\begin{equation}
    \average{x}=\frac{x_{1}+x_{2}+\ldots+x_{n}}{n}\equiv\frac{1}{n}\sum\limits _{i=1}^{n}x_{i}\label{eq:average}.
\end{equation}

Среднее арифметическое является далеко не единственным способом получения оценки по выборке (набору данных), в некоторых случаях к примеру используют наиболее вероятное значение, но такие случаи лежат за пределами данного пособия.

В теории вероятности показывается (см. раздел \ref{ch:prob}), что среднее арифметическое значение величин с одинаковым распределением имеет значительно меньший разброс, чем единичное измерение. Для количественной оценки разброса используют понятие \emph{дисперсии}. Понятие дисперсии и его формальное определение описаны в разделе \ref{ch:prob}. В этой главе ограничимся тем, что введем понятие среднеквадратичного отклонения (среднеквадратичное отклонение по определению равно корню из дисперсии), и его определение по серии измерений.

% Пусть число измерений велико: $n\gg1$. Если сказанное о стабильности
% прибора, методики и объекта измерения верно, можно ожидать, что сумма
% (\ref{eq:average}) будет стремиться к некоторому пределу\footnote{В теории вероятностей факт существования данного предела называют
% \emph{законом больших чисел}. Он выполняется при достаточно общих
% предположениях о характере случайной величины $x$. Тем не менее,
% в некоторых случаях этот предел может и не существовать. Такие ситуации
% требуют особого рассмотрения и мы на них не останавливаемся.}:
% \[
% \lim_{n\to\infty}\frac{1}{n}\sum_{i=1}^{n}x_{i}=x_{0}.
% \]
% Предельное значение $x_{0}$ можно условно назвать <<истинным>>
% средним для данной случайной величины.

Определим отклонение измерения от среднего как
\[
\Delta x_{i}=x_{i}-\average{x},\qquad i=1\ldots n.
\]
Теперь вычислим среднеквадратичное отклонение:
\begin{equation}
    s_{x}=\sqrt{\frac{\Delta x_{1}^{2}+\Delta x_{2}^{2}+\ldots+\Delta x_{n}^{2}}{n}}=\sqrt{\frac{1}{n}\sum\limits _{i=1}^{n}\Delta x_{i}^{2}}\label{eq:sigma}
\end{equation}
или кратко
\begin{equation}
    \boxed{s_{x}^{2}=\average{\left(x-\average{x}\right)^{2}}}.\label{eq:sigma_s}
\end{equation}
Квадрат $s_{x}^{2}$ называют выборочной дисперсией случайной величины. Предельную величину среднеквадратичного отклонения
при $n\to\infty$ обозначим как
\[
\sigma_{x}=\lim\limits _{n\to\infty}s_{x}.
\]

В математической статистике показывается, что если некоторая величина, распределенная по нормальному закону (см. \ref{sec:normal}) имеет среднее значение $\average{x}$ и стандартное отклонение $\sigma_x$, то при многократном повторении эксперимента приблизительно в 68\% случаев результат $x$ будет попадать в интервал $\average{x}-\sigma_{x}<x<\average{x}+\sigma_{x}.$


%{\footnotesize{}Отметим одну полезную формулу для дисперсии. Пользуясь
% тем, что среднее от суммы равно сумме средних, и раскрывая скобки
% в определении (\ref{eq:sigma_s}), найдём
% \begin{equation}
% s_{x}^{2}=\average{\left(x-\average{x}\right)^{2}}=\average{x^{2}-2x\average{x}+\average{x}^{2}}=\average{x^{2}}-\average{x}^{2},\label{eq:sigma_x2}
% \end{equation}
% то есть дисперсия равна разности среднего значения квадрата $\average{x^{2}}=\frac{1}{n}\sum\limits _{i=1}^{n}x_{i}^{2}$
% и квадрата среднего $\average{x}^{2}=\left(\frac{1}{n}\sum\limits _{i=1}^{n}x_{i}\right)^{2}$.}{\footnotesize\par}

\section{Классификация погрешностей}

Можно ли улучшить результаты опыта, повторяя его многократно? Будет
ли среднее значение $\average{x}$ <<лучше>>,
чем каждое измерение $x_{i}$ в отдельности? Чтобы ответить на эти
вопросы, проанализируем источники и виды погрешностей.

В первую очередь, многократное повторение опыта позволяет проверить
\emph{воспроизводимость} результатов: повторные измерения в \emph{одинаковых}
условиях, должны давать близкие результаты \footnote{Невоспроизводимые явления исследовать, очевидно, невозможно. Например,
мы почти ничего не знаем об устройстве <<шаровой молнии>>,
поскольку ещё никому не удалось создать стабильные условия для её
получения, хотя свидетельства существования этого редкого атмосферного
явления довольно многочисленны и надёжны.}\todo[author=Nozik]{Я предлагаю избавиться от footonote совсем. Они усложняют и читаемость и форматирование}. Таким образом, многократные измерения \emph{необходимы} для того,
чтобы убедиться как в надёжности методики, так и в существовании измеряемой
величины как таковой.

При любых измерениях возможны грубые ошибки --- \emph{промахи}
(\emph{англ.} miss). Это <<ошибки>> в стандартном
понимании этого слова --- возникающие по вине экспериментатора
или в силу других непредвиденных обстоятельств, например, из-за сбоя
аппаратуры. Промахов, конечно, нужно избегать, а результаты таких
измерений должны быть по возможности исключены из рассмотрения. Как
понять, является ли <<аномальный>> результат
промахом? Вопрос этот весьма непрост. В литературе существуют статистические
критерии отбора промахов, которыми мы, однако, настоятельно не рекомендуем
пользоваться (по крайней мере, без серьезного понимания последствий
такого отбора). Отбрасывание аномальных данных может, во-первых, привести
к тенденциозному искажению результата исследований, а во-вторых, так
можно упустить открытие неизвестного эффекта. Поэтому при научных
исследованиях необходимо максимально тщательно проанализировать причину
каждого промаха, в частности, многократно повторив эксперимент \footnote{Надо заметить, что это всё равно не даёт полной защиты от промахов,
ведь порой \emph{весь эксперимент} может оказаться одним большим <<промахом>>
(из недавних примеров можно вспомнить <<открытие>>
сверхсветовых нейтрино). Самый радикальный способ проверки достоверности
результатов --- многократная независимая проверка на других
приборах, другими методами, и желательно, другими экспериментаторами.}. Лишь только если факт и причина промаха установлены вполне достоверно,
соответствующий результат можно отбросить.

\note{
    Часто причины аномальных отклонений невозможно установить на этапе обработки данных, поскольку часть информации о проведении измерений к этому моменту утеряна. Единственным способ борьбы с этим - это очень подробное описание всего процесса измерений в \emph{лабораторном журнале}. Подробнее об этом написано в разделе \ref{sec:journal}.
}

При многократном повторении измерении одной и той же физической величины
погрешности могут иметь \emph{систематический} либо \emph{случайный}
характер. Назовём погрешность \emph{систематической}, если она повторяется
от опыта к опыту, сохраняя свой знак и величину, либо \emph{закономерно}
меняется в процессе измерений. \emph{Случайные} (или \emph{статистические})
погрешности меняются хаотично при повторении измерений как по величине,
так и по знаку, и в изменениях не прослеживается какой-либо закономерности.

Кроме того, удобно разделять погрешности по их происхождению. Можно
выделить
\begin{itemize}
    \item \emph{инструментальные} (или \emph{приборные}) \emph{погрешности},
связанные с несовершенством конструкции (неточности, допущенные при
изготовлении или вследствие старения), ошибками калибровки или ненормативными
условиями эксплуатации измерительных приборов;
    \item \emph{методические} \emph{погрешности}, связанные с несовершенством
теоретической модели явления (использование приближенных формул и
моделей явления) или с несовершенством методики измерения (например,
влиянием взаимодействия прибора и объекта измерения на результат измерения);
    \item \emph{естественные} \emph{погрешности}, связанные со случайным характером
измеряемой физической величины --- они являются не столько
<<ошибками>> измерения, сколько характеризуют
природу изучаемого объекта или явления.
\end{itemize}

\note{
    Разделение погрешностей на систематические и случайные
не является однозначным и зависит от постановки опыта. Например, производя
измерения не одним, а несколькими однотипными приборами, мы переводим
систематическую приборную ошибку, связанную с неточностью шкалы и
калибровки, в случайную. Разделение по происхождению также условно,
поскольку любой прибор подвержен воздействию <<естественных>>
случайных и систематических ошибок (шумы и наводки, тряска, атмосферные
условия и т.\,п.), а в основе работы прибора всегда лежит некоторое
физическое явление, описываемое не вполне совершенной теорией.
}

\subsubsection{Случайные погрешности}

Случайный характер присущ большому количеству различных физических
явлений и в той или иной степени проявляется в работе всех без исключения
приборов. Случайные погрешности обнаруживаются просто при многократном
повторении опыта --- в виде хаотичных изменений (\emph{флуктуаций})
значений $\left\{ x_{i}\right\} $.

Если случайные отклонения от среднего в большую или меньшую стороны
примерно равновероятны, можно рассчитывать, что при вычислении среднего
арифметического (\ref{eq:average}) эти отклонения скомпенсируются,
и погрешность результирующего значения $\average{x}$ будем меньше,
чем погрешность отдельного измерения.

Случайные погрешности бывают связаны, например,
\begin{enumerate}

    \item \emph{с особенностями используемых приборов}: техническими недостатками
(люфт в механических приспособлениях, сухое трение в креплении стрелки
прибора), с естественными (тепловой и дробовой шумы в электрических
цепях, тепловые флуктуации и колебания измерительных устройств из-за
хаотического движения молекул, космическое излучение) или техногенными
факторами (тряска, электромагнитные помехи и наводки);

    \item \emph{с особенностями и несовершенством методики измерения} (ошибка
при отсчёте по шкале, ошибка времени реакции при измерениях с секундомером);

    \item \emph{с несовершенством объекта измерений} (неровная поверхность,
неоднородность состава);

    \item \emph{cо случайным характером исследуемого явления} (радиоактивный
распад, броуновское движение).

\end{enumerate}

Остановимся несколько подробнее на двух последних случаях. Они отличаются
тем, что случайный разброс данных в них порождён непосредственно объектом
измерения. Если при этом приборные погрешности малы, то <<ошибка>>
эксперимента возникает лишь в тот момент, когда мы \emph{по своей
воле} совершаем замену ряда измеренных значений на некоторое среднее
$\left\{ x_{i}\right\} \to\average{x}$. Разброс данных при этом
характеризует не точность измерения, а сам исследуемый объект или
явление. Однако с \emph{математической} точки зрения приборные и <<естественные>>
погрешности \emph{неразличимы} --- глядя на одни только
экспериментальные данные невозможно выяснить, что именно явилось причиной
их флуктуаций: сам объект исследования или иные, внешние причины.
Таким образом, для исследования естественных случайных процессов необходимо
сперва отдельно исследовать и оценить случайные инструментальные погрешности
и убедиться, что они достаточно малы.

\note{
    Зависимость случайной погрешности среднего от погрешности отдельного измерения как правило имеет вид $\sigma_N = \sigma_1 / \sqrt{N}$ и выводится в разделе \ref{sec:average}.
}

\subsubsection{Систематические погрешности}

Систематические погрешности, в отличие от случайных, невозможно обнаружить,
исключить или уменьшить просто многократным повторением измерений.
Они могут быть обусловлены, во-первых, неправильной работой приборов
(\emph{инструментальная погрешность}), например, cдвигом нуля отсчёта
по шкале, деформацией шкалы, неправильной калибровкой, искажениями
из-за не нормативных условий эксплуатации, искажениями из-за износа
или деформации деталей прибора, изменением параметров прибора во времени
из-за нагрева и т.п. Во-вторых, их причиной может быть ошибка в интерпретации
результатов (\emph{методическая погрешность}), например, из-за использования
слишком идеализированной физической модели явления, которая не учитывает
некоторые значимые факторы (так, при взвешивании тел малой плотности
в атмосфере необходимо учитывать силу Архимеда; при измерениях в электрических
цепях может быть необходим учет неидеальности амперметров и вольтметров
и т.\,д.).

Систематические погрешности условно можно разделить на следующие категории.

\begin{enumerate}
    \item Известные погрешности, которые могут быть достаточно точно вычислены
или измерены. При необходимости они могут быть учтены непосредственно:
внесением поправок в расчётные формулы или в результаты измерений.
Если они малы, их можно отбросить, чтобы упростить вычисления.

    \item Погрешности известной природы, конкретная величина которых неизвестна,
но максимальное значение вносимой ошибки может быть определено теоретически
или экспериментально. Такие погрешности неизбежно присутствуют в любом
опыте, и задача экспериментатора --- свести их к минимуму,
совершенствуя методики измерения и выбирая более совершенные приборы.
    
    Чтобы оценить величину систематических погрешностей опыта, необходимо
учесть паспортную точность приборов (производитель, как правило, гарантирует,
что погрешность прибора не превосходит некоторой величины), проанализировать
особенности методики измерения, и по возможности, провести контрольные
опыты.

    \item Погрешности известной природы, оценка величины которых по каким-либо
причинам затруднена (например, сопротивление контактов при подключении
электронных приборов). Такие погрешности должны быть обязательно исключены
посредством модификации методики измерения или замены приборов.

    \item Наконец, нельзя забывать о возможности существования ошибок, о
которых мы не подозреваем, но которые могут существенно искажать результаты
измерений. Такие погрешности самые опасные, а исключить их можно только
многократной \emph{независимой} проверкой измерений, разными методами
и в разных условиях.
\end{enumerate}

В учебном практикуме учёт систематических погрешностей ограничивается,
как правило, паспортными погрешностями приборов и теоретическими поправками
к упрощенной модели исследуемого явления.
