\chapter{Элементы теории ошибок}
\label{ch:prob}

% Не претендуя ни в коей мере на полноту или строгость, мы изложим наиболее
% базовые понятия и результаты теории вероятностей, без которых нельзя обойтись
% при описании погрешностей.

Результат любого измерения не определён однозначно и имеет случайную составляющую.
Поэтому адекватным языком для описания погрешностей является язык вероятностей.
Тот факт, что значение некоторой величины \textquote{случайно}, не означает, что
она может принимать совершенно произвольные значения. Ясно, что частоты, с которыми
возникает те или иные значения, различны. Вероятностные законы, которым
подчиняются случайные величины, называют \emph{распределениями}.

\section{Случайная величина}

\emph{Случайной} будем называть величину, значение которой не может быть \emph{достоверно} определено экспериментатором. Чаще всего подразумевается, что случайная величина будет изменяться при многократном повторении одного и того же измерения. 
%При интерпретации результатов измерений в физических экспериментах, обычно 
Также случайной можно считать величину, значение которой фиксированно, но не известно экспериментатору (например, смещение нуля шкалы прибора). 

%Для формализации работы со случайными величинами используют понятие \emph{вероятности}. Численное значение вероятности того, что какая-то величина примет то или иное значение определяется либо как относительная частота наблюдения того или иного значения при повторении опыта большое количество раз, либо как оценка на основе данных других экспериментов.

Каждому из возможных значений некоторой случайной величины $x$ можно поставить
в соответствие значение \emph{вероятности} $P_x$ получить это значение при измерении.
Численно вероятность равна относительной частоте наблюдения этого значения, 
если бы опыт был повторён большое число раз:
 \[
 P_x = \lim_{n\to \infty} \frac{n_x}{n},
 \]
где $n$ --- полное число измерений, $n_x$ --- количество измерений, дающих результат~$x$. На практике значения $P_x$ могут быть получены как при многократном повторении опыта, либо как оценка на основе данных других экспериментов
или теоретической модели.

%\note{
%    Хотя понятия вероятности и случайной величины являются основополагающими, в литературе нет единства в их определении. Обсуждение формальных тонкостей или построение строгой теории лежит за пределами данного пособия. Поэтому на начальном этапе достаточно \textquote{интуитивного} представления. Заинтересованным читателям рекомендуем обратиться к специальной литературе: \cite{idie}.
%}

Большинство физических величин могут при измерениях принимать \emph{непрерывный} набор значений. Пусть $P_{[x_0,\,x_0+\delta x]}$~--- вероятность того, что результат
измерения величины~$x$ окажется вблизи некоторой точки $x_0$ в пределах интервала $\delta x$: $x\in [x_0,\,x_0+\delta x]$.
Устремим интервал
$\delta x$ к нулю. Нетрудно понять, что вероятность попасть в этот интервал
также будет стремиться к нулю (никакой результат измерения нельзя получить с абсолютной точностью!). Однако отношение
$w(x_0) = \frac{P_{[x_0,\,x_0+\delta x]}}{\delta x}$ будет оставаться конечным.
Функцию $w(x)$ называют \emph{плотностью распределения вероятности} или кратко
\emph{распределением} непрерывной случайной величины~$x$.

\note{В математической литературе распределением часто называют не функцию
    $w(x)$, а её интеграл $W(x)=\int w(x)\,dx$. Такую функцию в физике принято
    называть \emph{интегральным} или \emph{кумулятивным} распределением. В англоязычной литературе
    для этих функций принято использовать сокращения:
    \emph{pdf} (\emph{probability density function}) и
    \emph{cdf} (\emph{cumulative distribution function})
    соответственно.}

\paragraph{Гистограммы.}
Проиллюстрируем наглядно понятие плотности распределения. Результат
большого числа измерений случайной величины удобно представить с помощью
специального типа графика --- \emph{гистограммы}.
Для этого область значений~$x$, размещённую на оси абсцисс, разобьём на
равные малые интервалы --- \textquote{корзины} или \textquote{бины} (\emph{англ.} bins)
некоторого размера~$h$. По оси ординат будем откладывать долю измерений~$w$,
результаты которых попадают в соответствующую корзину. А именно,
пусть $k$~--- номер корзины; $n_k$~--- число измерений, попавших
в диапазон $x\in [kh,\,(k+1)h]$. Тогда  на графике изобразим \textquote{столбик}
шириной~$h$ и высотой $w_{k}=n_{k}/n$.
В результате получим картину, подобную изображённой на рис.~\ref{fig:normhist}.

\begin{figure}[ht!]
    \centering
    \includegraphics[width=9cm]{images/normhist.pdf}
    \caption{Пример гистограммы для нормального распределения ($\limaverage{x}=10$,
$\sigma=1{,}0$, $h=0{,}1$, $n=10^{4}$)}\label{fig:normhist}
\end{figure}

Согласно данному выше определению, высоты построенных столбиков будут приближённо соответствовать значению плотности распределения $w(x)$ вблизи соответствующей точки~$x$.
Если устремить число измерений к бесконечности ($n\to \infty$), а ширину корзин
к нулю ($h\to0$), то \emph{огибающая} гистограммы будет стремиться к некоторой
непрерывной функции $w(x)$.

Самые высокие столбики гистограммы будут группироваться вблизи максимума
функции $w(x)$ --- это \emph{наиболее вероятное} значение случайной величины.
Если отклонения в положительную и отрицательную стороны равновероятны,
то гистограмма будет симметрична --- в таком случае среднее значение $\average{x}$
также будет лежать вблизи этого максимума. Ширина гистограммы будет характеризовать разброс
значений случайной величины --- по порядку величины
она близка к среднеквадратичному отклонению~$s_x$.

\paragraph{Свойства распределений.}

Из определения функции $w(x)$ следует, что вероятность получить в результате
эксперимента величину $x$ в диапазоне от $a$ до $b$
% или вероятностное содержание интервала $(a,b)$
можно найти, вычислив интеграл:
\begin{equation}
    P_{x\in [a, b]}=\int\limits _{a}^{b}w\!\left(x\right)dx.\label{eq:P}
\end{equation}

Согласно определению вероятности, сумма вероятностей для всех возможных случаев
всегда равна единице. Поэтому интеграл распределения $w(x)$ по всей области
значений $x$ (то есть суммарная площадь под графиком $w(x)$) равен единице:
\[
\int\limits_{-\infty}^{+\infty} w(x)\,dx=1.
\]
Это соотношение называют \emph{условием нормировки}.

\paragraph{Среднее и дисперсия.}

Вычислим среднее по построенной гистограмме. Если размер корзин
$h$ достаточно мал, все измерения в пределах одной корзины можно считать примерно
одинаковыми. Тогда среднее арифметическое всех результатов можно вычислить как
\[
\average{x} \approx \frac{1}{n}\sum_i n_i x_i = \sum_i w_i x_i.
\]
Переходя к пределу, получим следующее определение среднего значения
случайной величины:
\begin{equation}
    \limaverage{x} = \int{x w\,dx},
\end{equation}
где интегрирование ведётся по всей области значений $x$.
В теории вероятностей \limaverage{x} также называют \emph{математическим ожиданием}
распределения.
Величину
\begin{equation}
    \sigma^2 = \limaverage{(x-\limaverage{x})^2}= \int{(x - \limaverage{x})^2 w\,dx}
\end{equation}
называют \emph{дисперсией} распределения. Значение $\sigma$ есть
срекднеквадратичное отклонение в пределе $n\to \infty$. Оно имеет ту
же размерность, что и сама величина $x$ и характеризует разброс распределения.
Именно эту величину, как правило, приводят как характеристику погрешности
измерения $x$.

\paragraph{Доверительный интервал.}
Обозначим как $P_{\left|\Delta x\right|<\delta}$ вероятность
того, что отклонение от среднего $\Delta x=x-\limaverage{x}$ составит величину,
не превосходящую по модулю значение $\delta$:
\begin{equation}\label{eq:confidenceP}
P_{\left|\Delta x\right|<\delta}=\int\limits
_{\limaverage{x}-\delta}^{\limaverage{x}+\delta}w\!\left(x\right)dx.
\end{equation}
Эту величину называют \emph{доверительной вероятностью} для
\emph{доверительного интервала} $\left|x-\limaverage{x}\right|\le\delta$.

% \todo{Что иллюстрирует пример?}
% \example{Пусть есть набор резисторов из одной серии с одной и той же маркировкой,
% соответствующей одному и тому же номиналу их сопротивления. В силу неидеальности
% процесса изготовления, точное значение резистора является случайной величиной,
% средней значение которое должно совпадать с заводским номиналом, но при этом
% будет наблюдаться также некоторый разброс значений. Рассмотрим два измерения: в
% первом будем просто брать все резисторы из одной серии, замерять их
% сопротивления и строить гистограмму результата измерений. В этом случае мы
% увидим картину похожую на рис.~\ref{fig:normhist}.\par
%     Теперь предположим, что перед тем как мы начали проводить свои измерения,
% кто-то взял и отобрал из изучаемой партии все резисторы с сопротивлением
% максимально приближенным к номинальному значению. В этом случае окажется, что в
% нашем измерении вероятность получить сопротивление, близкое к номиналу, будет
% мала. Как следствие, в результатем измерения будет получено \textquote{двух-горбое}
% распределение с провалом посередине. Среднее значение будет таким же, как и в
% первом случае, но разброс будет больше и наиболее вероятное значение (точнее два
% наиболее вероятных значения) не будут совпадать со средним.}


% Предположим, что систематические погрешности малы и займёмся отдельно
% изучением случайных погрешностей. Пусть по результатам многократных
% измерений получен набор значений $\left\{ x_{i}\right\} $, вычислено
% их среднее (\ref{eq:average}) $\average{x}$ и среднеквадратичное
% отклонение (\ref{eq:sigma}) $\sigma_{x}\approx s{}_{x}$. Можно надеяться,
% что измеряемая величина лежит в диапазоне
% \[
% x\in\left(\average{x}-\sigma_{x};\,\average{x}+\sigma_{x}\right).
% \]
% Какова вероятность $P$ того, что результат действительно находится
% в указанном интервале?

% Для ответа на этот вопрос необходимо знать \emph{вероятностный закон},
% которому подчиняется исследуемая величина. Казалось бы, для каждой
% случайной физической величины должен существовать свой особенный закон
% и общую теорию здесь построить невозможно. Это отчасти верно, но оказывается,
% что существует вполне \emph{универсальный} вероятностный закон, называемый
% \emph{нормальным}, которому подчиняются многие случайные величины.
% Рассмотрим его подробнее.

\section{Нормальное распределение}\label{sec:normal}

Одним из наиболее примечательных результатов теории вероятностей является
так называемая \emph{центральная предельная теорема}. Она утверждает,
что сумма большого количества независимых (см. ниже п.~\ref{sec:independent}) случайных слагаемых, каждое
из которых вносит в эту сумму относительно малый вклад, подчиняется
универсальному закону, не зависимо от того, каким вероятностным законам
подчиняются её составляющие. Это распределение называют \emph{нормальным}
или \emph{распределением Гаусса}.

Доказательство теоремы довольно громоздко и мы его не приводим. Остановимся
кратко на том, что такое нормальное распределение и его основных свойствах.

Плотность нормального распределения непрерывной случайной величины~$x$ выражается следующей формулой:
\begin{equation}
    \label{eq:normal}
    \boxed{
w_{\mathcal{N}}\!\left(x\right)=\frac{1}{\sqrt{2\pi}\sigma}e^{-\tfrac{(x-\limaverage{x})^
{2}}{2\sigma^{2}}}
    }.
\end{equation}
Здесь $\limaverage{x}$ и $\sigma$
--- параметры нормального распределения: $\limaverage{x}$ равно
среднему значению величины (математическому ожиданию), a $\sigma$ ---
её среднеквадратичному отклонению от среднего.

Функция $w_{\mathcal{N}}(x)$ представлена на рис.~\ref{fig:normhist}.
Распределение представляет собой симметричный
\textquote{колокол}, положение вершины которого
соответствует $\bar{x}$ (ввиду симметрии оно же
совпадает с наиболее вероятным значением --- максимумом
функции $w_{\mathcal{N}}(x)$).
При значительном отклонении $x$ от среднего величина
$w_{\mathcal{N}}\!\left(x\right)$
очень быстро убывает. Это означает, что вероятность встретить отклонения,
существенно большие, чем $\sigma$, оказывается пренебрежимо
мала. Ширина \textquote{колокола} по порядку величины
равна $\sigma$ --- она характеризует \textquote{разброс}
экспериментальных данных относительно среднего значения.

%\note{Точки $x=\bar{x}\pm\sigma$ являются точками
%    перегиба графика $w\left(x\right)$ (в них вторая производная по $x$
%    обращается в нуль, $w''=0$), а их положение по высоте составляет
%    $w\!\left(\bar{x}\pm\sigma\right)/w(\bar{x})=e^{_{-1/2}}\approx0{,}61$
%    от высоты вершины.}

Универсальный характер центральной предельной теоремы позволяет широко
применять на практике нормальное (гауссово) распределение для обработки
результатов измерений, поскольку в большинстве случаев случайные погрешности складываются из
множества случайных \emph{независимых} факторов. Заметим, что на практике
для \emph{приближённой оценки} параметров нормального распределения
случайной величины используются \emph{выборочные} значения среднего
и дисперсии: $\limaverage{x}\approx\average{x}$, $\sigma_{x}\approx s_{x}$.

\begin{figure}[ht]
    \centering
    \input{images/gauss.pdf_t}
    \caption{Плотность нормального распределения}
\end{figure}

\paragraph{Доверительные вероятности.}
Вычислим некоторые доверительные вероятности \eqref{eq:confidenceP} для нормально
распределённых случайных величин.

Вероятность того, что результат отдельного измерения $x$ окажется
в пределах $\limaverage{x}\pm\sigma$, равна площади под графиком
функции $w_{\mathcal{N}}(x)$ в данном интервале:
\[
P_{\left|\Delta x\right|<\sigma} =
\int\limits_{\limaverage{x}-\sigma}^{\limaverage{x}+\sigma}
w_{\mathcal{N}} dx \approx0{,}68.
\]
\note{
    Интеграл вида $\int e^{-x^{2}}dx$, называемый \emph{интегралом ошибок}, в  элементарных функциях не выражается, но легко находится численными методами. Соответствующая функция, обычно обозначаемая как \texttt{erf}, реализована большинстве математических программных пакетов.
}
Вероятность отклонения в пределах $\limaverage{x}\pm2\sigma$:
\[
P_{\left|\Delta x\right|<2\sigma}\approx0{,}95,
\]
а в пределах $\limaverage{x}\pm3\sigma$:
\[
P_{\left|\Delta x\right|<3\sigma}\approx0{,}9973.
\]
Иными словами, при большом числе измерений нормально распределённой
величины можно ожидать, что лишь треть измерений выпадут за пределы интервала
$\left[\bar{x}-\sigma,\,\bar{x}+\sigma\right]$. При этом около 5\%
измерений выпадут за пределы $\left[\bar{x}-2\sigma;\bar{x}+2\sigma\right]$,
и лишь 0,27\% окажутся за пределами
$\left[\bar{x}-3\sigma;\bar{x}+3\sigma\right]$.

\example{В сообщениях об открытии бозона Хиггса на Большом адронном коллайдере
в 2012 году говорилось о том, что исследователи ждали подтверждение результатов
со  \textquote{статистической значимостью 5 сигма}. Используя нормальное распределение \eqref{eq:normal}
нетрудно посчитать, что они использовали доверительную вероятность
$P\approx1-5{,}7\cdot10^{-7}=0{,}99999943$. Такую достоверность можно назвать фантастической!}

Полученные значения доверительных вероятностей используются при
\emph{стандартной записи результатов измерений}. В физических измерениях, как правило, используется $P=0{,}68$,
то есть, запись
\[
x=\bar{x}\pm\delta x
\]
означает, что измеренное значение лежит в диапазоне (доверительном
интервале) $x\in\left[\bar{x}-\delta x;\bar{x}+\delta x\right]$ с
вероятностью 68\%. Таким образом погрешность $\pm\delta x$ считается
равной одному среднеквадратичному отклонению: $\delta x=\sigma$.
В \emph{технических} измерениях чаще используется $P=0{,}95$, то есть под
абсолютной погрешностью имеется в виду удвоенное среднеквадратичное
отклонение, $\delta x=2\sigma$. Во избежание разночтений доверительную
вероятность следует указывать отдельно.

\note{Хотя нормальный закон распределения встречается на практике довольно часто, стоит помнить, что он реализуется \emph{не всегда}.
Нарушение полученных соотношений для долей измерений, попадающих в соответствующие интервалы можно использовать как признак 
\textquote{нормальности}
исследуемого распределения.}

\paragraph{Сравнение результатов измерений.}
Теперь мы можем дать количественный критерий для сравнения двух измеренных
величин или двух результатов измерения одной и той же величины.

Пусть $x_{1}$ и $x_{2}$ ($x_{1}\ne x_{2}$) измерены с
погрешностями $\sigma_{1}$ и $\sigma_{2}$ соответственно.
Ясно, что если различие результатов $|x_2-x_1|$ невелико,
его можно объяснить просто случайными отклонениями.
Если же теоретическая вероятность обнаружить такое отклонение
достаточно мала, различие результатов следует признать \emph{значимым}.

Граничное значение вероятности, в принципе, может быть выбрано произвольным
образом. Наиболее часто в качестве границы выбирают вероятность $P=5\%$,
что для нормального распределения соответствует двум стандартным отклонениям.

Допустим сначала, что одна из величин известна с существенно большей точностью:
$\sigma_{2}\ll\sigma_{1}$ (например, $x_{1}$ --- результат, полученный
студентом в лаборатории, $x_{2}$ --- справочное значение).
Поскольку $\sigma_{2}$ мало, $x_{2}$ можно принять за \textquote{истинное}:
$x_{2}\approx \limaverage{x}$. Предполагая, что погрешность измерения
$x_{1}$ подчиняется нормальному закону с и дисперсией $\sigma_{1}^{2}$,
можно утверждать, что
% можно с помощью функции (\ref{eq:normal}) вычислить вероятность
% того, что отклонение $\left|x_{1}-x_{2}\right|$ возникло исключительно
% в силу случайных причин.
% То есть
различие считают будет значимы, если разность результатов превышает
удвоенное значение погрешности: $\left|x_{1}-x_{2}\right|>2\sigma_{1}$.

Пусть теперь погрешности измерений сравнимы по порядку величины:
$\sigma_{1}\sim\sigma_{2}$. В теории вероятностей показывается, что
линейная комбинация нормально распределённых величин также имеет нормальное
распределение с дисперсией $\sigma^{2}=\sigma_{1}^{2}+\sigma_{2}^{2}$
(см. также правила сложения погрешностей (\ref{eq:sigma_sum})). Тогда
для проверки гипотезы о том, что $x_{1}$ и $x_{2}$ являются измерениями
одной и той же величины, нужно вычислить, является ли значимым отклонение
$\left|x_{1}-x_{2}\right|$ от нуля при $\sigma=\sqrt{\sigma_{1}^{2}+\sigma_{2}^{2}}$.

\example{Два студента получили следующие значения для теплоты испарения
    некоторой жидкости: $x_{1}=40{,}3\pm0{,}2$~кДж/моль и
    $x_{2}=41{,}0\pm0{,}3$~кДж/моль, где погрешность соответствует
    одному стандартному отклонению. Можно ли утверждать, что они исследовали
    одну и ту же жидкость?\par
Имеем наблюдаемую разность $\left|x_{1}-x_{2}\right|=0{,}7$~кДж/моль,
среднеквадратичное отклонение для разности
$\sigma=\sqrt{0{,}2^{2}+0{,}3^{2}}=0{,}36$~кДж/моль.
Их отношение $\frac{\left|x_{2}-x_{1}\right|}{\sigma}\approx2$. Из
свойств нормального распределения находим: вероятность того, что измерялась
одна и та же величина, а различия в ответах возникли из-за случайных
ошибок, равна $P\approx5\%$. 
%Ответ на вопрос, \textquote{достаточно}
%ли мала или велика эта вероятность, остаётся на усмотрение исследователя.
}

%\note{Изложенные здесь соображения применимы, только если $\limaverage{x}$ и
%его стандартное отклонение $\sigma$ получены на основании достаточно
%большой выборки $n\gg1$ (или заданы точно). При небольшом числе измерений
%($n\lesssim10$) выборочные средние $\average{x}$ и среднеквадратичное отклонение
%$s_x$ сами имеют довольно большую погрешность, а
%их распределение будет описываться не нормальным законом, а так
%называемым $t$-распределением Стъюдента. В частности, в зависимости от
%значения $n$ интервал $\average{x}\pm s_{x}$ будет соответствовать несколько
%меньшей доверительной вероятности, чем $P=0{,}68$. Особенно резко различия
%проявляются при высоких уровнях доверительных вероятностей $P\to1$. 
%Подробнее см. \cite{idie}.
% \todo[inline, author = Nozik]{Все слова про распределение Стьюдента сильно путают и студентов и преподавателей. В лучшем случае, все студенты знают, что нужно что-то домножить на коэффициент стьюдента, но не знают, что и почему. Если вообще вводить это здесь, то нужно говорить, что именно распределено по этому распределению. Ну и ссылка нужна.}
%}

% \section{Распределение Пуассона}
% \disclaimer{
%     Распределение Пуассона применяется в случаях, когда имеет место измерения
% количества событий, произошедших за определенный интервал времени или в
% определенном объеме. Понимание этого распределения необходимо для студентов 5
% семестра, изучающих основы физички частиц. Остальные студенты могут пропустить
% этот раздел.
% }
%
% \todo[inline, color = red]{TODO Пуассон}

% \todo[inline,author=ppv]{Пусть корреляции будут в приложении, но какие-то слова о независимости величин сказать надо}

\section{Независимые величины}\label{sec:independent}
Величины $x$ и $y$ называют \emph{независимыми} если результат измерения одной 
из них никак не влияет на результат измерения другой. Для таких величин вероятность того, что  $x$ примет значения из некоторого множества $X$, и одновременно $y$ --- в множестве $Y$, 
равна произведению соответствующих вероятностей:
\[
P_{x\in X , y\in Y} = P_{x\in X}\cdot P_{y\in Y}.
\]

Обозначим отклонения величин от их средних как $\Delta x=x-\limaverage{x}$ и 
$\Delta y=y-\limaverage{y}$.
Средние значения этих отклонений равны, очевидно, нулю: 
$\limaverage{\Delta
x}=\limaverage{x}-\limaverage{x}=0$, $\limaverage{\Delta y}=0$.
Из независимости величин~$x$ и~$y$ следует,
что среднее значение от произведения 
$\limaverage{\Delta x\cdot\Delta y}$
равно произведению средних $\limaverage{\Delta x}\cdot\limaverage{\Delta y}$
и, следовательно, равно нулю:
\begin{equation}
\limaverage{\Delta x\cdot\Delta y}=\limaverage{\Delta x}\cdot\limaverage{\Delta
y}=0.\label{eq:indep}
\end{equation}

Пусть измеряемая величина $z=x+y$ складывается из двух \emph{независимых}
случайных слагаемых $x$ и $y$, для которых известны  средние 
$\limaverage{x}$ и $\limaverage{y}$, и их среднеквадратичные погрешности
$\sigma_{x}$ и $\sigma_{y}$. Непосредственно из определения (\ref{eq:average})
следует, что среднее суммы равно сумме средних:
\[
    \limaverage{z}=\limaverage{x}+\limaverage{y}.
\]

Найдём дисперсию $\sigma_{z}^{2}$. В силу независимости имеем
\[
    \limaverage{\Delta z^{2}}=\limaverage{\Delta x^{2}}+\limaverage{\Delta
    y^{2}}+2\limaverage{\Delta x\cdot\Delta y} = \limaverage{\Delta x^{2}}+
    \limaverage{\Delta    y^{2}},
\]
то есть:
\begin{equation}
    \label{eq:sigma_sum}
    \sigma_z^2=\sigma_{x}^{2}+\sigma_{y}^{2}.
\end{equation}
Таким образом, при сложении \emph{независимых} величин их погрешности
складываются \emph{среднеквадратичным} образом.

Подчеркнём, что для справедливости соотношения (\ref{eq:sigma_sum})
величины $x$ и $y$ не обязаны быть нормально распределёнными ---
достаточно чтобы их дисперсии были конечны. Однако можно
показать, что если $x$ и $y$ распределены нормально, \emph{нормальным
будет и распределение их суммы}.

\enlargethispage{1em}

\note{Требование независимости
слагаемых является принципиальным. Например, положим $y=x$. Тогда
$z=2x$. Здесь $y$ и $x$, очевидно, зависят друг от друга. Используя
(\ref{eq:sigma_sum}), находим $\sigma_{2x}=\sqrt{2}\sigma_{x}$,
что, конечно, неверно --- непосредственно из определения
следует, что $\sigma_{2x}=2\sigma_{x}$.}

Отдельно стоит обсудить математическую структуру формулы (\ref{eq:sigma_sum}).
Если одна из погрешностей много больше другой, например,
$\sigma_{x}\gg\sigma_{y}$,
то меньшей погрешностью можно пренебречь, $\sigma_{x+y}\approx\sigma_{x}$.
С другой стороны, если два источника погрешностей имеют один порядок
$\sigma_{x}\sim\sigma_{y}$, то и $\sigma_{x+y}\sim\sigma_{x}\sim\sigma_{y}$.
Эти обстоятельства важны при планирования эксперимента: как правило,
величина, измеренная наименее точно, вносит наибольший вклад в погрешность
конечного результата. При этом, пока не устранены наиболее существенные
ошибки, бессмысленно гнаться за повышением точности измерения остальных
величин.

\example{Пусть $\sigma_{y}=\sigma_{x}/3$,
тогда $\sigma_{z}=\sigma_{x}\sqrt{1+\frac{1}{9}}\approx1{,}05\sigma_{x}$,
то есть при различии двух погрешностей более, чем в 3 раза, поправка
к погрешности составляет менее 5\%, и уже нет особого смысла в учёте
меньшей погрешности: $\sigma_{z}\approx\sigma_{x}$. Это утверждение
касается сложения любых \emph{независимых} источников погрешностей в эксперименте.}

\section{Погрешность среднего}\label{sec:average}

Выборочное среднее арифметическое значение $\average{x}$, найденное
по результатам $n$ измерений, само является случайной величиной.
Действительно, если поставить серию одинаковых опытов по $n$ измерений,
то в каждом опыте получится своё среднее значение, отличающееся от
предельного среднего $\limaverage{x}$.

Вычислим среднеквадратичную погрешность среднего арифметического
$\sigma_{\average{x}}$.
Рассмотрим вспомогательную сумму $n$ слагаемых
\[
    Z=x_{1}+x_{2}+\ldots+x_{n}.
\]
Если $\left\{ x_{i}\right\} $ есть набор \emph{независимых} измерений
\emph{одной и той же} физической величины, то мы можем, применяя результат
(\ref{eq:sigma_sum}) предыдущего параграфа, записать
\[
    \sigma_{Z}=\sqrt{\sigma_{1}^{2}+\sigma_{2}^{2}+\ldots+\sigma_{n}^{2}
    }=\sqrt{n}\sigma_{x},
\]
поскольку под корнем находится $n$ одинаковых слагаемых. Отсюда с
учётом $\average{x}=Z/n$ получаем важное соотношение:
\begin{equation}
\boxed{{\sigma_{\average{x}}=\frac{\sigma_{x}}{\sqrt{n}}}}.\label{eq:sigma_avg}
\end{equation}
Таким образом, \emph{погрешность среднего значения $x$ по результатам~$n$ независимых измерений оказывается в $\sqrt{n}$ раз меньше погрешности
отдельного измерения}. Именно этот факт позволяет
уменьшать случайные погрешности эксперимента за счёт многократного
повторения измерений.

Подчеркнём различия между $\sigma_{x}$ и $\sigma_{\average{x}}$:

величина $\sigma_{x}$ --- \emph{погрешность отдельного
измерения} --- является характеристикой разброса значений
в совокупности измерений $\left\{ x_{i}\right\} $, $i=1..n$. При
нормальном законе распределения примерно $P=68\%$ измерений попадают в
интервал $\average{x}\pm\sigma_{x}$;

величина $\sigma_{\average{x}}$ --- \emph{погрешность
среднего} --- характеризует точность, с которой определено
среднее значение измеряемой физической величины $\average{x}$ относительно
предельного (\textquote{истинного}) среднего $\limaverage{x}$;
при этом с доверительной вероятностью $P=68\%$ искомая величина $\limaverage{x}$
лежит в интервале
$\average{x}-\sigma_{\average{x}}<\limaverage{x}<\average{x}+\sigma_{\average{x}}$.


\section{Результирующая погрешность опыта}

Пусть для некоторого результата измерения известна оценка его максимальной
систематической погрешности $\Delta_{\text{сист}}$ и случайная
среднеквадратичная
погрешность $\sigma_{\text{случ}}$. Какова \textquote{полная}
погрешность измерения?

Предположим для простоты, что измеряемая величина \emph{в принципе}
может быть определена сколь угодно точно, так что можно говорить о
некотором её \textquote{истинном} значении $x_{\text{ист}}$
(иными словами, погрешность результата связана в основном именно с
процессом измерения). Назовём \emph{полной погрешностью} измерения
среднеквадратичное значения отклонения от результата измерения от
\textquote{истинного}:
\[
\sigma_{\text{полн}}^{2}=\average{\left(x-x_{\text{ист}}\right)^{2}}.
\]
Отклонение $x-x_{\text{ист}}$ можно представить как сумму случайного
отклонения от среднего $\delta x_{\text{случ}}=x-\limaverage{x}$
и постоянной (но, вообще говоря, неизвестной) систематической составляющей
$\delta x_{\text{сист}}= \limaverage{x} - x_{\text{ист}} = \mathrm{const}$:
\[
x-x_{\text{ист}}=\delta x_{\text{сист}}+\delta x_{\text{случ}}.
\]
Причём случайную составляющую можно считать независимой от систематической.
В таком случае аналогично \eqref{eq:sigma_sum} находим:
\begin{equation}
\sigma_{\text{полн}}^{2}=\average{\delta x_{\text{сист}}^{2}}+\average{\delta
x_{\text{случ}}^{2}}\le\Delta_{\text{сист}}^{2}+\sigma_{\text{случ}}^{2}.
\label{eq:syst_full}
\end{equation}
То есть для получения оценки значения полной
погрешности некоторого измерения нужно квадратично сложить максимальную
систематическую и случайную погрешности.

\note{Согласно данному нами в начале главы определению,
    неизвестное значение систематической погрешности также можно считать
    случайной величиной (например, мы пользуемся линейкой, при изготовлении которой на заводе произошло некоторое случайное искажение шкалы). В~такой
    трактовке формула \eqref{eq:syst_full} есть просто
    частный случай \eqref{eq:sigma_sum}.\par
    Подчеркнем однако, что вероятностный закон, которому подчиняется
    систематическая ошибка, как правило \emph{неизвестен}. 
%    Поэтому неизвестно и
%    распределение итогового результата. Отсюда, в частности, следует,
    Следовательно, мы, строго говоря, не можем приписать интервалу $x\pm\Delta_{\text{сист}}$ какую-либо
    определённую доверительную вероятность.
%     (она равна 0,68
%    только если систематическая ошибка имеет нормальное распределение).
%    Можно, конечно, \emph{предположить},
%    что ошибки (например, при изготовлении линеек на заводе) имеют гауссов характер. Также часто полагают, что систематическая ошибка имеет \emph{равномерное}
%    распределение (то есть \textquote{истинное} значение может с равной вероятностью
%    принять любое значение в пределах интервала $\pm\Delta_{\text{сист}}$).
%    Оба предположения могут быть в той или иной степени разумны, но зачастую ни для одного из них нет достаточных оснований.
}

Если измерения проводятся многократно, то согласно (\ref{eq:sigma_avg})
случайная составляющая погрешности может быть уменьшена, а систематическая
составляющая при этом остаётся неизменной:
\[
\sigma_{\text{полн}}^{2}\le\Delta_{\text{сист}}^{2}+\frac{\sigma_{x}^{2}}{n}.
\]

Отсюда следует важное практическое правило
(см. также обсуждение в п.~\ref{sec:independent}): если случайная погрешность измерений
в 2--3 раза меньше предполагаемой систематической, то
\emph{нет смысла проводить многократные измерения} в попытке уменьшить погрешность
всего эксперимента. В такой ситуации измерения достаточно повторить
2--3 раза --- чтобы убедиться в повторяемости результата, исключить промахи
и проверить, что случайная ошибка действительно мала.
В противном случае повторение измерений может иметь смысл до
тех пор, пока погрешность среднего
$\sigma_{\average{x}}=\frac{\sigma_{x}}{\sqrt{n}}$
не станет меньше систематической.



\example{В результате измерения диаметра проволоки микрометром,
имеющим цену деления $h=0,01$ мм, получен следующий набор из $n=8$ значений:\par
{\footnotesize
\begin{tabular}{|c|c|c|c|c|c|c|c|c|}
\hline
$d$, мм & 0,39 & 0,38 & 0,39 & 0,37 & 0,40 & 0,39 & 0,38 & 0,39 \\ \hline
\end{tabular}\par}
\smallskip
Вычисляем среднее значение: $\average{d}\approx386{,}3$~мкм.
Среднеквадратичное (стандартное) отклонение:
% вычисляем по формуле (\ref{eq:sigma_straight}):
$\sigma_{d}\approx9{,}2$~мкм. Случайная погрешность среднего согласно
(\ref{eq:sigma_avg}):
$\sigma_{\average{d}}=\frac{\sigma_{d}}{\sqrt{8}}\approx3{,}2$~мкм. Все результаты лежат в пределах $\pm2\sigma_{d}$, поэтому нет
причин сомневаться в нормальности распределения. Максимальную погрешность
микрометра оценим как половину цены деления, $\Delta=h/2=5$~мкм.
Результирующая полная погрешность
$\sigma\le\sqrt{\Delta^{2}+\frac{\sigma_{d}^{2}}{8}}\approx6{,}0$~мкм.
Видно, что $\sigma_{\text{случ}}\approx\Delta_{\text{сист}}$ и проводить дополнительные измерения особого смысла нет. Окончательно результат измерений может быть представлен в виде (см. также \emph{правила округления}
результатов измерений в п.~\ref{subsec:round})
\[
d=386\pm6\;\text{мкм},\qquad\varepsilon_{d}=1{,}5\%.
\]

Заметим, что поскольку случайная погрешность и погрешность
прибора здесь имеют один порядок величины, наблюдаемый случайный разброс
данных может быть связан как с неоднородностью сечения проволоки,
так и с дефектами микрометра (например, с неровностями зажимов, люфтом
винта, сухим трением, деформацией проволоки под действием микрометра
и т.\,п.). Для ответа на вопрос, что именно вызвало разброс, требуются
дополнительные исследования с использованием более точных
приборов.\par
}%\footnotesize

\example{Измерение скорости
полёта пули было осуществлено с погрешностью $\delta v=\pm1$ м/c.
Результаты измерений для $n=6$ выстрелов представлены в таблице:\par
{\footnotesize
\begin{tabular}{|c|c|c|c|c|c|c|}
\hline
$v$, м/с & 146 & 170 & 160 & 181 & 147 & 168 \\ \hline
\end{tabular}\par}
\smallskip
Усреднённый результат $\average{v}=162{,}0\;\text{м/с}$,
стандартное отклонение $\sigma_{v}=13{,}8\;\text{м/c}$, случайная
ошибка для средней скорости
$\sigma_{\bar{v}}=\sigma_{v}/\sqrt{6}=5{,}6\;\text{м/с}$.
Поскольку разброс экспериментальных данных существенно превышает погрешность
каждого измерения, $\sigma_{v}\gg\delta v$, он почти наверняка связан
с реальным различием скоростей пули в разных выстрелах, а не с ошибками
измерений. В качестве результата эксперимента представляют интерес
как среднее значение скоростей $\average{v}=162\pm6\;\text{м/с}$
($\varepsilon\approx4\%$), так и значение $\sigma_{v}\approx14\;\text{м/с}$,
характеризующее разброс значений скоростей от выстрела к выстрелу.
Малая инструментальная погрешность в принципе позволяет более точно
измерить среднее и дисперсию, и исследовать закон распределения выстрелов
по скоростям более детально --- для этого требуется набрать
б\'{о}льшую статистику по выстрелам.\par
}%\footnotesize

\example{Измерение скорости
полёта пули было проведено с погрешностью $\delta v=10$ м/c. Результаты
измерений для $n=6$ выстрелов:\par
{\footnotesize
\begin{tabular}{|c|c|c|c|c|c|c|}
\hline
$v$, м/с & 150 & 170 & 160 & 180 & 150 & 170 \\ \hline
\end{tabular}\par}
\smallskip
Усреднённый результат $\average{v}=163{,}3\;\text{м/с}$,
$\sigma_{v}=12{,}1\;\text{м/c}$, $\sigma_{\average{v}}=5\;\text{м/с}$,
$\sigma_{\text{полн}}\approx11{,}2\;\text{м/с}$. Инструментальная
погрешность каждого измерения превышает разброс данных, поэтому в
этом опыте затруднительно сделать вывод о различии скоростей от выстрела
к выстрелу. Результат измерений скорости пули:
$\average{v}=163\pm11\;\text{м/с}$,
$\varepsilon\approx7\%$. Проводить дополнительные выстрелы при такой
большой инструментальной погрешности особого смысла нет ---
лучше поработать над точностью приборов и методикой измерений.\par
}%\footnotesize


\section{Обработка косвенных измерений\label{sec:kosv}}

\emph{Косвенными} называют измерения, полученные в результате расчётов,
использующих результаты \emph{прямых} (то есть \textquote{непосредственных})
измерений физических величин. 
%Сформулируем основные правила пересчёта
%погрешностей при косвенных измерениях.

\subsection{Случай одной переменной}

Пусть в эксперименте измеряется величина $x$, а её \textquote{наилучшее}
(в некотором смысле) значение равно $x^{\star}$ и оно известно с
погрешностью $\sigma_{x}$. После чего с помощью известной функции
вычисляется величина $y=f\!\left(x\right)$.

В качестве \textquote{наилучшего} приближения для $y$ используем значение функции
при \textquote{наилучшем} $x$:
\[
y^{\star}=f\!\left(x^{\star}\right).
\]

Найдём величину погрешности $\sigma_{y}$. Обозначая отклонение измеряемой
величины как $\Delta x=x-x^{\star}$, и пользуясь определением производной,
при условии, что функция $y\left(x\right)$ --- гладкая
вблизи $x\approx x^{\star}$, запишем
\[
\Delta y\equiv y\left(x\right)-y\left(x^{\star}\right)\approx f'\cdot\Delta x,
\]
где $f'\equiv\frac{dy}{dx}$ --- производная фукнции $f(x)$, взятая в точке
$x^{\star}$. Возведём полученное в квадрат, проведём усреднение
($\sigma_{y}^{2}=\average{\Delta y^{2}}$,
$\sigma_{x}^{2}=\average{\Delta x^{2}}$), и затем снова извлечём
корень. В результате получим
\begin{equation}
{\sigma_{y}=\left|\frac{dy}{dx}\right|\sigma_{x}.}\label{eq:sxy}
\end{equation}

\example{Для степенной функции
$y=Ax^{n}$ имеем $\sigma_{y}=nAx^{n-1}\sigma_{x}$, откуда
\[
\frac{\sigma_{y}}{y}=n\frac{\sigma_{x}}{x},\qquad\text{или}\qquad\varepsilon_{y}
=n\varepsilon_{x},
\]
то есть относительная погрешность степенной функции возрастает пропорционально
показателю степени $n$.\par
}%\footnotesize

\example{Для $y=1/x$ имеем $\varepsilon_{1/x}=\varepsilon_{x}$
--- при обращении величины сохраняется её относительная
погрешность.\par
}%\footnotesize

\exercise{Найти погрешность $\sigma_y$ логарифма $y=\ln x$ по заданным
$x$ и $\sigma_x$.}

\exercise{Найдите погрешность $\sigma_y$ показательной функции $y=a^{x}$,
    если известны~$x$ и~$\sigma_{x}$. Коэффициент $a$ задан точно.}

\subsection{Случай многих переменных}

Пусть величина $u$ вычисляется по измеренным значениям нескольких
различных \emph{независимых} физических величин $x$, $y$, $\ldots$
на основе известного закона $u=f\!\left(x,y,\ldots\right)$. В качестве
наилучшего значения можно по-прежнему взять значение функции $f$
при наилучших значениях измеряемых параметров:
\[
u^{\star}=f\!\left(x^{\star},y^{\star},\ldots\right).
\]

Для нахождения погрешности $\sigma_{u}$ воспользуемся свойством,
известным из математического анализа, --- малые приращения гладких
функции многих переменных складываются линейно, то есть справедлив
\emph{принцип суперпозиции} малых приращений:
\[
\Delta u\approx f'_{x}\cdot\Delta x+f'_{y}\cdot\Delta y+\ldots,
\]
где символом $f'_{x}\equiv\frac{\partial f}{\partial x}$ обозначена
\emph{частная производная} функции $f$ по переменной $x$ ---
то есть обычная производная $f$ по $x$, взятая при условии, что
все остальные аргументы (кроме $x$) считаются постоянными параметрами.
Тогда пользуясь формулой для нахождения дисперсии суммы независимых
величин (\ref{eq:sigma_sum}), получим соотношение, позволяющее вычислять
погрешности косвенных измерений для произвольной функции
$u=f\left(x,y,\ldots\right)$:
\begin{equation}
\boxed{\sigma_{u}^{2}=f_{x}^{\prime2}\,\sigma_{x}^{2}+f_{y}^{\prime2}\,\sigma_{y
}^{2}+\ldots}\label{eq:sigma_general}
\end{equation}
%Это и есть искомая общая формула пересчёта погрешностей при косвенных
%измерениях.

Отметим, что формулы (\ref{eq:sxy}) и (\ref{eq:sigma_general}) применимы
только если относительные отклонения всех величин малы
($\varepsilon_{x},\varepsilon_{y},\ldots\ll1$),
а измерения проводятся вдали от особых точек функции $f$ (производные
$f_{x}'$, $f_{y}'$ $\ldots$ не должны обращаться в бесконечность).
Также подчеркнём, что все полученные здесь формулы справедливы только
для \emph{независимых} переменных $x$, $y$, $\ldots$

Остановимся на некоторых важных частных случаях формулы
(\ref{eq:sigma_general}).

\example{Для суммы (или разности) $u=\sum\limits _{i=1}^{n}a_{i}x_{i}$ имеем
\begin{equation}
\sigma_{u}^{2}=\sum_{i=1}^{n}a_{i}^{2}\sigma_{x_{i}}^{2}.
\end{equation}}

\example{Найдём погрешность степенной функции     $u=x^{\alpha}\cdot y^{\beta}\cdot\ldots$:
\[
\frac{\sigma_{u}^{2}}{u^{2}}=\alpha^{2}\frac{\sigma_{x}^{2}}{x^{2}}+\beta^{2}
\frac{\sigma_{y}^{2}}{y^{2}}+\ldots
\]
или через относительные погрешности
\begin{equation}
\varepsilon_{u}^{2}=\alpha^{2}\varepsilon_{x}^{2}+\beta^{2}\varepsilon_{y}^{2}
+\ldots\label{eq:espilon_power}
\end{equation}}

\example{Вычислим погрешность произведения и частного $u=xy$ или $u=x/y$.
    Тогда в обоих случаях имеем
\begin{equation}
\varepsilon_{u}^{2}=\varepsilon_{x}^{2}+\varepsilon_{y}^{2},
\end{equation}
то есть при умножении или делении относительные погрешности складываются
квадратично.}

%\example{Рассмотрим несколько более сложный случай: нахождение угла по его тангенсу
%\[
%u=\arctg\frac{y}{x}.
%\]
%В таком случае, пользуясь тем, что $\left(\arctg z\right)'=\frac{1}{1+z^{2}}$,
%где $z=y/x$, и используя производную сложной функции, находим
%$u_{x}'=u_{z}'z'_{x}=-\frac{y}{x^{2}+y^{2}}$,
%$u_{y}'=u'_{z}z'_{y}=\frac{x}{x^{2}+y^{2}}$, и наконец
%\[
%\sigma_{u}^{2}=\frac{y^{2}\sigma_{x}^{2}+x^{2}\sigma_{y}^{2}}{\left(x^{2}+y^{2}
%\right)^{2}}.
%\]}

\exercise{Найти погрешность $\sigma_z$ вычисления гипотенузы $z=\sqrt{x^{2}+y^{2}}$
    прямоугольного треугольника по измеренным катетам $x$ и $y$.}

\exercise{Найти погрешность $\sigma_{\alpha}$ вычисления угла прямоугольного треугольника по его катетам:
$\alpha=\arctg\frac{y}{x}$.}

Исходя из полученных результатов можно дать следующие практические рекомендации.
\begin{itemize}\small
\item Как правило, нет смысла увеличивать точность измерения какой-то одной величины, если другие величины, используемые в расчётах, остаются
измеренными относительно грубо --- всё равно итоговая погрешность
будет определяться самым неточным измерением. 
Поэтому все измерения следует проводить \emph{примерно с одной и той же относительной погрешностью}.
\item При этом, как следует из (\ref{eq:espilon_power}), особое внимание следует уделять величинам, возводимым при расчётах в степени с большими показателями. 
\item При сложных функциональных зависимостях
имеет смысл детально проанализировать структуру формулы
(\ref{eq:sigma_general}):
если вклад от некоторой величины в общую погрешность мал, нет смысла
гнаться за высокой точностью её измерения, и наоборот, точность некоторых измерений может оказаться критически важной 
(особенно осторожным нужно быть на участках резкого
изменения функции $f$, когда производная $|f'_x|$ велика).
\item Следует избегать измерения малых величин как разности двух близких
значений (например, толщины стенки цилиндра как разности внутреннего
и внешнего радиусов): если $u=x-y$, то абсолютная погрешность
$\sigma_{u}=\sqrt{\sigma_{x}^{2}+\sigma_{y}^{2}}$
меняется мало, однако относительная погрешность
$\varepsilon_{u}=\frac{\sigma_{u}}{x-y}$
может оказаться неприемлемо большой, если $x\approx y$.
\end{itemize}

